\hypertarget{ux3c0ux3c1ux3bfux3c0ux3c4ux3c5ux3c7ux3b9ux3b1ux3baux3b5ux3c3-ux3c3ux3c0ux3bfux3c5ux3b4ux3b5ux3c3}{%
\chapter{ΠΡΟΠΤΥΧΙΑΚΕΣ
ΣΠΟΥΔΕΣ}\label{ux3c0ux3c1ux3bfux3c0ux3c4ux3c5ux3c7ux3b9ux3b1ux3baux3b5ux3c3-ux3c3ux3c0ux3bfux3c5ux3b4ux3b5ux3c3}}

\hypertarget{ux3b5ux3b9ux3c3ux3b1ux3b3ux3c9ux3b3ux3ae}{%
\section{Εισαγωγή}\label{ux3b5ux3b9ux3c3ux3b1ux3b3ux3c9ux3b3ux3ae}}

Το Τμήμα Πληροφορικής προσφέρει ένα προπτυχιακό πρόγραμμα διάρκειας 4
ακαδημαϊκών ετών (οκτώ εξάμηνα σπουδών), το οποίο παρέχει στους
αποφοίτους του την επιστημονική γνώση και την πρακτική εξάσκηση που
απαιτούνται για να ανταποκριθούν στη σύγχρονη αγορά εργασίας του κλάδου
της Πληροφορικής.

\hypertarget{ux3bfux3bcux3acux3b4ux3b5ux3c2-ux3bcux3b1ux3b8ux3b7ux3bcux3acux3c4ux3c9ux3bd-ux3baux3b1ux3c4ux3b5ux3c5ux3b8ux3cdux3bdux3c3ux3b5ux3b9ux3c2}{%
\section{Ομάδες Μαθημάτων
(κατευθύνσεις)}\label{ux3bfux3bcux3acux3b4ux3b5ux3c2-ux3bcux3b1ux3b8ux3b7ux3bcux3acux3c4ux3c9ux3bd-ux3baux3b1ux3c4ux3b5ux3c5ux3b8ux3cdux3bdux3c3ux3b5ux3b9ux3c2}}

Τα μαθήματα χωρίζονται σε τρεις κατηγορίες: κορμού (υποχρεωτικά για
όλους τους φοιτητές), κατεύθυνσης (υποχρεωτικά για τους φοιτητές που
έχουν επιλέξει την συγκεκριμένη κατεύθυνση) και επιλογής (διαθέσιμα προς
επιλογή για τους φοιτητές και των δυο κατευθύνσεων). Τα δύο πρώτα έτη (4
εξάμηνα) σπουδών, οι φοιτητές παρακολουθούν μαθήματα κορμού και
επιλογής. Οι φοιτητές/φοιτήτριες επιλέγουν την κατεύθυνση (ομάδα
γνωστικών αντικειμένων και αντίστοιχων μαθημάτων) στην οποία
ειδικεύονται, αποκτώντας με αυτό τον τρόπο μία εις βάθος γνώση και
εμπειρία. Το Τμήμα προσφέρει τις εξής κατευθύνσεις:

\begin{itemize}
\tightlist
\item
  Πληροφορική -- Ανθρωπιστικές και Κοινωνικές Επιστήμες
\item
  Πληροφοριακά Συστήματα
\end{itemize}

Οι φοιτητές επιλέγουν μια από τις δυο κατευθύνσεις σπουδών με δήλωσή
τους στην Γραμματεία. Για την ενημέρωση των φοιτητών/τριών σχετικά με
τις κατευθύνσεις στην αρχή του 5ου εξαμήνου γίνεται μια ενημερωτική
παρουσίαση των κατευθύνσεων. Η επιλογή κατεύθυνσης πραγματοποιείται μέσα
στις πρώτες δυο διδακτικές εβδομάδας του χειμερινού εξαμήνου του τρίτου
έτους (5ου εξαμήνου). Ένας φοιτητής δύναται να πραγματοποιήσει αλλαγή
στην επιλογή κατεύθυνσης με δήλωσή του στη Γραμματεία, μία μόνο φορά,
όποτε επιθυμεί. Η επιλογή της κατεύθυνσης αναγράφεται στην αναλυτική
βαθμολογία του φοιτητή, ενώ δεν αναγράφεται στο πτυχίο.

\hypertarget{ux3baux3b1ux3bdux3bfux3bdux3b9ux3c3ux3bcux3ccux3c2-ux3c0ux3c1ux3bfux3c0ux3c4ux3c5ux3c7ux3b9ux3b1ux3baux3ceux3bd-ux3c3ux3c0ux3bfux3c5ux3b4ux3ceux3bd}{%
\section{Κανονισμός Προπτυχιακών
Σπουδών}\label{ux3baux3b1ux3bdux3bfux3bdux3b9ux3c3ux3bcux3ccux3c2-ux3c0ux3c1ux3bfux3c0ux3c4ux3c5ux3c7ux3b9ux3b1ux3baux3ceux3bd-ux3c3ux3c0ux3bfux3c5ux3b4ux3ceux3bd}}

\hypertarget{ux3c7ux3c1ux3bfux3bdux3b9ux3baux3ae-ux3b4ux3b9ux3acux3c1ux3b8ux3c1ux3c9ux3c3ux3b7-ux3c4ux3bfux3c5-ux3c0ux3c1ux3bfux3c0ux3c4ux3c5ux3c7ux3b9ux3b1ux3baux3bfux3cd-ux3c0ux3c1ux3bfux3b3ux3c1ux3acux3bcux3bcux3b1ux3c4ux3bfux3c2-ux3c3ux3c0ux3bfux3c5ux3b4ux3ceux3bd}{%
\subsection{Χρονική Διάρθρωση του Προπτυχιακού Προγράμματος
Σπουδών}\label{ux3c7ux3c1ux3bfux3bdux3b9ux3baux3ae-ux3b4ux3b9ux3acux3c1ux3b8ux3c1ux3c9ux3c3ux3b7-ux3c4ux3bfux3c5-ux3c0ux3c1ux3bfux3c0ux3c4ux3c5ux3c7ux3b9ux3b1ux3baux3bfux3cd-ux3c0ux3c1ux3bfux3b3ux3c1ux3acux3bcux3bcux3b1ux3c4ux3bfux3c2-ux3c3ux3c0ux3bfux3c5ux3b4ux3ceux3bd}}

Οι σπουδές του Π.Π.Σ. διεξάγονται με το σύστημα των εξαμηνιαίων
μαθημάτων. Η διάρκεια του Π.Π.Σ. είναι οκτώ (8) εξάμηνα κατανεμημένα σε
τέσσερα (4) ακαδημαϊκά έτη. Το ακαδημαϊκό έτος αρχίζει την 1η
Σεπτεμβρίου και λήγει την 31η Αυγούστου του επόμενου έτους. Το διδακτικό
έργο κάθε ακαδημαϊκού έτους διαρθρώνεται χρονικά σε δύο (2) ακαδημαϊκά
εξάμηνα, το χειμερινό και το εαρινό, με τις ακριβείς ημερομηνίες έναρξης
και λήξης τους να καθορίζονται από το Ακαδημαϊκό Ημερολόγιο του
Ιδρύματος. Το χρονικό διάστημα διδασκαλίας για κάθε εξάμηνο περιλαμβάνει
κατ' ελάχιστο δεκατρείς (13) πλήρεις εβδομάδες εκπαιδευτικών
δραστηριοτήτων. Στο πλαίσιο των υποχρεωτικών μαθημάτων και των μαθημάτων
επιλογής του Προγράμματος Σπουδών, προβλέπονται ώρες θεωρίας, αλλά
επίσης προβλέπονται και ώρες διδασκαλίας για τα φροντιστηριακά και
εργαστηριακά μαθήματα και την εν γένει άσκηση των φοιτητών του Τμήματος.
Μετά το πέρας της διδασκαλίας κάθε εξαμήνου σπουδών ακολουθεί η εκάστοτε
εξεταστική περίοδος, ενώ κατά τον μήνα Σεπτέμβριο κάθε ακαδημαϊκού έτους
λαμβάνει χώρα επαναληπτική εξεταστική περίοδος (που αφορά στα μαθήματα
που έχουν διδαχθεί και στα δύο (2) εξάμηνα του προηγούμενου ακαδημαϊκού
έτους). Ο αριθμός των εβδομάδων για τη διενέργεια των εξετάσεων,
ορίζεται με απόφαση της Συγκλήτου και περιλαμβάνεται στο Ακαδημαϊκό
Ημερολόγιο του Ιδρύματος. Κατά τη διάρκεια των εξαμήνων, μαθήματα και
εξετάσεις δεν διεξάγονται τις επίσημες εθνικές και τοπικές αργίες

\hypertarget{ux3b5ux3b3ux3b3ux3c1ux3b1ux3c6ux3adux3c2-ux3bcux3b5ux3c4ux3b5ux3b3ux3b3ux3c1ux3b1ux3c6ux3adux3c2}{%
\subsection{Εγγραφές --
Μετεγγραφές}\label{ux3b5ux3b3ux3b3ux3c1ux3b1ux3c6ux3adux3c2-ux3bcux3b5ux3c4ux3b5ux3b3ux3b3ux3c1ux3b1ux3c6ux3adux3c2}}

Φοιτητές/ριες του Τμήματος Πληροφορικής του Ιονίου Πανεπιστημίου
καθίστανται όσοι/ες εγγράφονται σ' αυτό μετά από εισαγωγή, μετεγγραφή ή
κατατακτήριες εξετάσεις σύμφωνα με τις κείμενες διατάξεις. Η εγγραφή των
πρωτοετών φοιτητών/ριών στο Τμήμα Πληροφορικής γίνεται μέσω της
ηλεκτρονικής εφαρμογής https://eregister.it.minedu.gov.gr/ του
Υπουργείου Παιδείας και Θρησκευμάτων και αποστολή των ονομάτων των
επιτυχόντων από το Υπουργείο στο Τμήμα. Ακολουθεί η διαδικασία της
ταυτοποίησης των στοιχείων (ταυτοπροσωπία) των επιτυχόντων από τη
Γραμματεία του Τμήματος. Ενημέρωση των ενδιαφερομένων γίνεται από τη
Γραμματεία με ανάρτηση σχετικών ανακοινώσεων στην ιστοσελίδα του
Τμήματος. Η ταυτοπροσωπία είναι υποχρεωτική προκειμένου να ολοκληρωθεί η
εγγραφή και να χορηγηθούν στους φοιτητές κωδικοί πρόσβασης στις
υπηρεσίες του Πανεπιστημίου. Οι μετεγγραφές φοιτητών/ριών (αφορά μόνο τα
Τμήματα για τα οποία υπάρχει αντιστοιχία με το Τμήμα Πληροφορικής)
διενεργούνται από το Υπουργείο Παιδείας και Θρησκευμάτων στο οποίο
κατατίθενται ηλεκτρονικά οι αντίστοιχες αιτήσεις μετεγγραφής σε
ημερομηνίες που ανακοινώνει το Υπουργείο. Φοιτητής/ρια που έχει εγγραφεί
στο Τμήμα Πληροφορικής του Ιονίου Πανεπιστημίου δεν μπορεί να είναι
συγχρόνως φοιτητής/τρια και σε άλλο εκπαιδευτικό ίδρυμα τριτοβάθμιας
εκπαίδευσης.

\hypertarget{ux3b1ux3bdux3b1ux3c3ux3c4ux3bfux3bbux3ae-ux3c3ux3c0ux3bfux3c5ux3b4ux3ceux3bd}{%
\subsection{Αναστολή
Σπουδών}\label{ux3b1ux3bdux3b1ux3c3ux3c4ux3bfux3bbux3ae-ux3c3ux3c0ux3bfux3c5ux3b4ux3ceux3bd}}

Κάθε φοιτητής/ρια έχει δικαίωμα να ζητήσει αναστολή σπουδών με αίτησή
του στη Γραμματεία του Τμήματος Πληροφορικής (για όσα εξάμηνα,
συνεχόμενα ή μη, επιθυμεί, και πάντως όχι περισσότερα από 8, δηλαδή τον
ελάχιστο αριθμό εξαμήνων που απαιτούνται για τη λήψη πτυχίου σύμφωνα με
το πρόγραμμα σπουδών του Τμήματος Πληροφορικής. Η αίτηση περιλαμβάνει το
αιτούμενο χρονικό διάστημα αναστολής και είναι προαιρετική η αναφορά των
λόγων σε αυτή. Οι φοιτητές/ριες που διακόπτουν κατά τα ανωτέρω τις
σπουδές τους, δεν διατηρούν τη φοιτητική ιδιότητα καθ' όλο το χρονικό
διάστημα της διακοπής των σπουδών τους. Μετά τη λήξη της διακοπής, οι
φοιτητές επανέρχονται και εντάσσονται ξανά στο Τμήμα. Οι πρωτοετείς
φοιτητές/ριες υποβάλλουν αίτηση διακοπής φοίτησης εφόσον έχει
ολοκληρωθεί η διαδικασία αρχικής εγγραφής. Δεν έχουν δικαίωμα υποβολής
αίτησης διακοπής φοίτησης, εάν έχουν ήδη λάβει βεβαίωση σπουδών για το Α
́ εξάμηνο. Το συνολικό χρονικό διάστημα της αναστολής σπουδών δεν
προσμετράται στον υπολογισμό των ετών φοίτησης.

\hypertarget{ux3b4ux3b7ux3bbux3ceux3c3ux3b5ux3b9ux3c2-ux3bcux3b1ux3b8ux3b7ux3bcux3acux3c4ux3c9ux3bd}{%
\subsection{Δηλώσεις
Μαθημάτων}\label{ux3b4ux3b7ux3bbux3ceux3c3ux3b5ux3b9ux3c2-ux3bcux3b1ux3b8ux3b7ux3bcux3acux3c4ux3c9ux3bd}}

Οι φοιτητές/ριες εγγράφονται στο Τμήμα στην αρχή κάθε εξαμήνου, σε
ημερομηνίες που ορίζονται από την Κοσμητεία, ανακοινώνονται από τη
Γραμματεία και αναρτώνται στην ιστοσελίδα του Τμήματος δηλώνοντας
συγχρόνως τα μαθήματα που επιλέγουν να παρακολουθήσουν κατά το
συγκεκριμένο εξάμηνο (δήλωση μαθημάτων) υποβάλλοντας ηλεκτρονική δήλωση
μέσω του συστήματος https://dias.ionio.gr/. Όσοι φοιτητές/ριες δεν
δηλώσουν εμπρόθεσμα τα υποχρεωτικά και επιλογής μαθήματα, θεωρείται ότι
δεν έχουν ανανεώσει την εγγραφή τους για το τρέχον εξάμηνο και δεν έχουν
δικαίωμα συμμετοχής στις αντίστοιχες εξετάσεις, συμπεριλαμβανομένων και
αυτών του Σεπτεμβρίου. Μαθήματα τα οποία έχουν δηλωθεί σε προηγούμενα
ακαδημαϊκά έτη και δεν έχουν ολοκληρωθεί με επιτυχία, θα πρέπει να
δηλωθούν εκ νέου προκειμένου να δοθεί η δυνατότητα παρακολούθησης και
επανεξέτασης στο τρέχον ακαδημαϊκό έτος.

\hypertarget{ux3b5ux3c0ux3b9ux3bbux3bfux3b3ux3ae-ux3bcux3b1ux3b8ux3b7ux3bcux3acux3c4ux3c9ux3bd}{%
\subsection{Επιλογή
Μαθημάτων}\label{ux3b5ux3c0ux3b9ux3bbux3bfux3b3ux3ae-ux3bcux3b1ux3b8ux3b7ux3bcux3acux3c4ux3c9ux3bd}}

Δεν χρειάζεται αντικατάσταση επιλεγόμενου μαθήματος σε περίπτωση μη
επιτυχούς εξέτασης σε αυτό. Μάθημα επιλογής που είχε δηλωθεί σε
παλαιότερη Δήλωση του φοιτητή και δεν εξετάστηκε ή δεν εξετάστηκε
επιτυχώς, πρέπει είτε α) να δηλωθεί ξανά στην τρέχουσα Δήλωση αν ο
φοιτητής/φοιτήτρια θέλει να επανεξεταστεί είτε β) απλώς να μην δηλωθεί
ξανά στην τρέχουσα Δήλωση αν ο φοιτητής/φοιτήτρια δεν επιθυμεί να
επανεξεταστεί.

Οι φοιτητές/φοιτήτριες έχουν τη δυνατότητα να δηλώσουν ως επιλεγόμενο,
ένα υποχρεωτικό μάθημα άλλης κατεύθυνσης επιπλέον των μαθημάτων επιλογής
του εξαμήνου στο οποίο βρίσκονται.

Μόνο στο τρίτο (3ο) και τέταρτο (4ο) έτος σπουδών μπορούν οι
φοιτητές/φοιτήτριες να δηλώνουν αριθμό επιλεγομένων μαθημάτων που
υπερβαίνει τον καθορισμένο από το Πρόγραμμα Σπουδών. Οι
φοιτητές/φοιτήτριες δε μπορούν να δηλώσουν μαθήματα επιλογής μεγαλύτερων
εξαμήνων του Τμήματος ή άλλων Τμημάτων του Ιονίου Πανεπιστημίου.

Τα μαθήματα ελεύθερης επιλογής ΕΕ (δηλαδή μαθήματα επιλογής που όμως
διδάσκονται από άλλα Τμήματα του Ιονίου Πανεπιστημίου) πρέπει α) να
αντιστοιχούν σε τρεις (3) ή περισσότερες διδακτικές μονάδες στο
Πρόγραμμα Σπουδών του άλλου Τμήματος και β) το περιεχόμενό τους να μην
είναι Πληροφορική αφού αυτό το περιεχόμενο καλύπτεται πλήρως στο Τμήμα
Πληροφορικής, αλλά να σχετίζεται με τα αντικείμενα της Πληροφορικής.
Μόνον ένα μάθημα Ελεύθερης Επιλογής μπορεί να υπολογιστεί στη διαμόρφωση
του βαθμού του πτυχίου.

Για να δηλώσει ο φοιτητής/η φοιτήτρια μάθημα ελεύθερης επιλογής ΕΕ από
άλλο Τμήμα, θα πρέπει πρώτα να έρθει σε συνεννόηση με τη Γραμματεία του
Τμήματος Πληροφορικής.

Εάν ένας φοιτητής/φοιτήτρια έχει περάσει περισσότερα από τα απαιτούμενα
για τη λήψη πτυχίου επιλεγόμενα μαθήματα μπορεί μόνο με έγγραφη δήλωσή
του να καθορίσει τα μαθήματα τα οποία θα διαμορφώσουν το βαθμό του
πτυχίου του.

\hypertarget{ux3c0ux3c1ux3bfux3cbux3c0ux3bfux3b8ux3adux3c3ux3b5ux3b9ux3c2-ux3b3ux3b9ux3b1-ux3c4ux3b7ux3bd-ux3b1ux3c0ux3ccux3baux3c4ux3b7ux3c3ux3b7-ux3c0ux3c4ux3c5ux3c7ux3afux3bfux3c5}{%
\subsection{Προϋποθέσεις για την απόκτηση
πτυχίου}\label{ux3c0ux3c1ux3bfux3cbux3c0ux3bfux3b8ux3adux3c3ux3b5ux3b9ux3c2-ux3b3ux3b9ux3b1-ux3c4ux3b7ux3bd-ux3b1ux3c0ux3ccux3baux3c4ux3b7ux3c3ux3b7-ux3c0ux3c4ux3c5ux3c7ux3afux3bfux3c5}}

Προϋπόθεση για την απόκτηση πτυχίου είναι η πιστοποιημένη γνώση της
Αγγλικής γλώσσας που αποδεικνύεται με κατάθεση κρατικού πιστοποιητικού
γλωσσομάθειας τουλάχιστον επιπέδου Β2 ή άλλου αντίστοιχου, ή με επιτυχή
εξέταση στα μαθήματα των Αγγλικών, ή με βεβαίωση παρακολούθησης
σεμιναρίων Αγγλικής γλώσσας για την περίπτωση που οργανωθούν από το
Τμήμα σεμινάρια με τον σκοπό αυτό.

Για την απόκτηση πτυχίου απαιτείται η επιτυχής συγγραφή της Πτυχιακής
Εργασίας, η επιτυχής γραπτή δοκιμασία σε όλα τα υποχρεωτικά μαθήματα (Υ)
και σε τόσα μαθήματα επιλογής όσα χρειάζονται ώστε το άθροισμα των
μονάδων ECTS (European Credit Transfer and Accumulation System) των
υποχρεωτικών μαθημάτων, των μονάδων ECTS των μαθημάτων επιλογής, και των
μονάδων ECTS που αντιστοιχούν στην Πτυχιακή Εργασία, να είναι
τουλάχιστον 240 ECTS. Τα προσφερόμενα μαθήματα αντιστοιχούν σε
τουλάχιστον 60 μονάδες ECTS ανά ακαδημαϊκό έτος.

Οι διδακτικές μονάδες για κάθε μάθημα είναι τέσσερις (4), οπότε ο
συντελεστής βαρύτητας στην διαμόρφωση του βαθμού πτυχίου είναι ενάμιση
(1,5). (Δεν πρέπει να συγχέονται οι «Διδακτικές Μονάδες» με τις «ECTS
Μονάδες»).

Η συγγραφή της Πτυχιακής Εργασίας ισοδυναμεί με δύο μαθήματα επιλογής
που το καθένα αντιστοιχεί σε 6 μονάδες ECTS. Επομένως η Πτυχιακή Εργασία
αντιστοιχεί συνολικά σε 12 μονάδες ECTS. Ο συντελεστής βαρύτητας της
Πτυχιακής Εργασίας στην διαμόρφωση του βαθμού πτυχίου είναι τέσσερα (4).
Γλώσσα συγγραφής της Πτυχιακής Εργασίας είναι η ελληνική αποκλειστικά. Ο
τίτλος της Πτυχιακής Εργασίας επίσης πρέπει να είναι στην ελληνική
γλώσσα. Προαιρετικά είναι θεμιτή η ύπαρξη Περίληψης της Πτυχιακής
Εργασίας στην αγγλική γλώσσα.

Η δήλωση θέματος και τριμελούς συμβουλευτικής επιτροπής για την Πτυχιακή
Εργασία είναι δυνατή μόνον αν πληρούνται ορισμένες προϋποθέσεις οι
οποίες αναφέρονται στον Κανονισμό Πτυχιακών Εργασιών. Οι προϋποθέσεις
αυτές αφορούν μαθήματα του πρώτου και δεύτερου έτους, γι' αυτό είναι
χρήσιμο ο φοιτητής να τις γνωρίζει από την αρχή των σπουδών του. Ο
Κανονισμός Πτυχιακών Εργασιών βρίσκεται αναρτημένος στην ιστοσελίδα του
Τμήματος.

Η Πρακτική Άσκηση είναι προαιρετική, συνολικής διάρκειας δύο μηνών.
Σημειώνεται ότι η δήλωση της πρακτικής άσκησης ως μαθήματος επιλογής
γίνεται υπό προϋποθέσεις σύμφωνα με τον Κανονισμό Πρακτικής Άσκησης. Οι
ECTS μονάδες της Πρακτικής Άσκησης ορίζονται κάθε χρόνο με απόφαση του
υπουργείου Παιδείας. Αυτή τη στιγμή είναι 8 ECTS, αλλά μπορεί να
αλλάξουν με νεότερη απόφαση. Η βαθμολογία στο μάθημα της Πρακτικής
Άσκησης έχει τη μορφή «επιτυχώς/ανεπιτυχώς» και δεν συμμετέχει στην
διαμόρφωση του βαθμού πτυχίου.

Προϋπόθεση για την απόκτηση πτυχίου είναι η πιστοποιημένη γνώση της
Αγγλικής γλώσσας που αποδεικνύεται με κατάθεση κρατικού πιστοποιητικού
γλωσσομάθειας τουλάχιστον επιπέδου Β2 ή άλλου αντίστοιχου, ή με επιτυχή
εξέταση στα μαθήματα των Αγγλικών, ή με βεβαίωση παρακολούθησης
σεμιναρίων Αγγλικής γλώσσας για την περίπτωση που οργανωθούν από το
Τμήμα σεμινάρια με τον σκοπό αυτό.

Η τελική βαθμολογία του πτυχίου υπολογίζεται ως ο σταθμισμένος μέσος
όρος των επιμέρους βαθμολογιών των μαθημάτων και της πτυχιακής εργασίας
του ενδεικτικού προγράμματος σπουδών του Τμήματος, χωρίς να λαμβάνεται
υπόψη η πρακτική άσκηση. Ο τελικός υπολογισμός αναγράφεται με ακρίβεια
δεύτερου δεκαδικού ψηφίου. Η κλίμακα βαθμολογίας του πτυχίου έχει ως
εξής:

\begin{itemize}
\tightlist
\item
  Άριστα: από 8,50 έως 10
\item
  Λίαν Καλώς: από 6,50 έως 8,49
\item
  Καλώς: Από 5 έως 6,49.
\end{itemize}

\hypertarget{ux3b5ux3beux3b5ux3c4ux3acux3c3ux3b5ux3b9ux3c2-ux3b1ux3beux3b9ux3bfux3bbux3ccux3b3ux3b7ux3c3ux3b7-ux3c6ux3bfux3b9ux3c4ux3b7ux3c4ux3ceux3bd}{%
\subsection{Εξετάσεις -- Αξιολόγηση
Φοιτητών}\label{ux3b5ux3beux3b5ux3c4ux3acux3c3ux3b5ux3b9ux3c2-ux3b1ux3beux3b9ux3bfux3bbux3ccux3b3ux3b7ux3c3ux3b7-ux3c6ux3bfux3b9ux3c4ux3b7ux3c4ux3ceux3bd}}

Οι εξετάσεις διενεργούνται αποκλειστικά μετά το πέρας του χειμερινού και
του εαρινού εξαμήνου για τα μαθήματα που διδάχθηκαν στα εξάμηνα αυτά,
αντίστοιχα. Η έναρξη και λήξη των εξεταστικών περιόδων περιλαμβάνονται
στο Ακαδημαϊκό Ημερολόγιο. Οι φοιτητές δικαιούνται να εξεταστούν στα
μαθήματα και των δύο εξαμήνων σε επαναληπτική εξέταση που διενεργείται
το μήνα Σεπτέμβριο.

Οι φοιτητές που περάτωσαν την κανονική φοίτηση, η οποία ισούται με τον
ελάχιστο αριθμό των αναγκαίων για την απονομή του τίτλου σπουδών
εξαμήνων, σύμφωνα με το ενδεικτικό πρόγραμμα σπουδών, και επιπλέον
οφείλουν έως έναν συγκεκριμένο αριθμό μαθημάτων, έχουν τη δυνατότητα να
εξεταστούν στην εξεταστική περίοδο του χειμερινού και του εαρινού
εξαμήνου κάθε ακαδημαϊκού έτους στα μαθήματα που οφείλουν, ανεξάρτητα
εάν αυτά διδάσκονται σε χειμερινό ή εαρινό εξάμηνο, έπειτα από απόφαση
της Συνέλευσης του Τμήματος.

Φοιτητής που δεν παρακολούθησε με επιτυχία υποχρεωτικό μάθημα πρέπει να
το επαναλάβει. Αν απέτυχε σε μάθημα επιλογής, μπορεί να το επαναλάβει ή
να το αντικαταστήσει με άλλο μάθημα επιλογής.

Οι φοιτητές όλων των εξαμήνων μπορούν με έγγραφη δήλωση να δηλώσουν
μέχρι πέντε (5) μαθήματα συνολικά στη διάρκεια των σπουδών τους και μία
μόνο φορά ανά μάθημα, για επανεξέταση με σκοπό την βελτίωση βαθμολογίας,
σύμφωνα με τη διαδικασία του άρθρου 36 του Κανονισμού Λειτουργίας
Ιδρύματος του Ιονίου Πανεπιστημίου.

Οι φοιτητές όλων των εξαμήνων που έχουν εξεταστεί ανεπιτυχώς
περισσότερες από 3 (τρεις) φορές σε μάθημα, μπορούν με αίτησή τους να
ζητήσουν αναβαθμολόγηση γραπτού, σύμφωνα με τη διαδικασία του άρθρου 36
του Κανονισμού Λειτουργίας Ιδρύματος του Ιονίου Πανεπιστημίου.

Βεβαίωση συμμετοχής στις εξετάσεις δικαιούνται μόνο οι φοιτητές που
έχουν δηλώσει το μάθημα και εξετάζονται σε αυτό. Η βεβαίωση παρέχεται
από τη Γραμματεία έπειτα από τη διενέργεια των αναγκαίων διασταυρώσεων
με τον/τη διδάσκοντα/διδάσκουσα που έχει την ευθύνη της εξέτασης του
μαθήματος.

Δεν καταχωρούνται βαθμοί για ονόματα φοιτητών/τριών που δεν
συμπεριλαμβάνονται στο φύλλο παρουσίας των εξεταζόμενων ή που δεν έχουν
δηλώσει το μάθημα.

Οι φοιτητές έχουν δικαίωμα μετά το πέρας των εξετάσεων και της ανάρτησης
των αποτελεσμάτων σε ειδικά καθορισμένες μέρες και ώρες που
ανακοινώνονται από τον/την διδάσκοντα/ουσα του μαθήματος να βλέπουν το
γραπτό τους και να ζητούν διευκρινήσεις για τον τρόπο που αυτό
αξιολογήθηκε.

Στους φοιτητές/τριες οι οποίοι/ες προσκομίζουν στη Γραμματεία του
Τμήματος φοίτησής τους ειδικές διαγνωστικές εκθέσεις -- ως αυτές
ορίζονται από το εκάστοτε νομοθετικό πλαίσιο -- παρέχονται όλες οι
προσήκουσες, σύμφωνα με τις προσκομιζόμενες εκθέσεις και τη νομοθεσία,
προσαρμογές των τρόπων εξέτασης για την πληρέστατη δυνατή προσβασιμότητα
της εκπαιδευτικής διαδικασίας.

\hypertarget{ux3b1ux3bdux3b1ux3b3ux3bdux3ceux3c1ux3b9ux3c3ux3b7-ux3bcux3b1ux3b8ux3b7ux3bcux3acux3c4ux3c9ux3bd}{%
\subsection{Αναγνώριση
Μαθημάτων}\label{ux3b1ux3bdux3b1ux3b3ux3bdux3ceux3c1ux3b9ux3c3ux3b7-ux3bcux3b1ux3b8ux3b7ux3bcux3acux3c4ux3c9ux3bd}}

Οι φοιτητές/φοιτήτριες που έχουν εισαχθεί στο Τμήμα με κατατακτήριες
εξετάσεις ή με διαδικασία μετεγγραφής δύνανται, με αίτησή τους στην
Γραμματεία, να ζητήσουν την αναγνώριση ενός η περισσοτέρων μαθημάτων του
Π.Π.Σ. Η αίτηση αναγνώρισης πρέπει να συνοδεύεται από α) αναλυτική
βαθμολογία του φοιτητή από τις σπουδές του στο Τμήμα Προηγούμενου
Πτυχίου και β) οδηγό σπουδών του Τμήματος Προηγούμενου Πτυχίου που να
περιλαμβάνει αναλυτική περιγραφή του περιεχομένου των μαθημάτων που
επιθυμεί ο φοιτητής να ληφθούν υπόψη κατά την αναγνώριση.
Φοιτητές/φοιτήτριες που συμμετέχουν στο πρόγραμμα ERASMUS+ ανταλλαγής
φοιτητών/τριών μπορούν να ζητήσουν την αναγνώριση μαθημάτων που έχουν
επιλέξει στο Πανεπιστήμιο του εξωτερικού, βάσει του Κανονισμού ERASMUS+
του Τμήματος και του Ιδρύματος.

\hypertarget{ux3baux3b1ux3c4ux3b1ux3c4ux3b1ux3baux3c4ux3aeux3c1ux3b9ux3b5ux3c2-ux3b5ux3beux3b5ux3c4ux3acux3c3ux3b5ux3b9ux3c2}{%
\subsection{Κατατακτήριες
Εξετάσεις}\label{ux3baux3b1ux3c4ux3b1ux3c4ux3b1ux3baux3c4ux3aeux3c1ux3b9ux3b5ux3c2-ux3b5ux3beux3b5ux3c4ux3acux3c3ux3b5ux3b9ux3c2}}

Η επιλογή των υποψηφίων για κατάταξη πτυχιούχων τριτοβάθμιας εκπαίδευσης
στο Τμήμα για την απόκτηση δεύτερου πτυχίου γίνεται αποκλειστικά με
κατατακτήριες εξετάσεις με θέματα ανάπτυξης σε τρία (3) μαθήματα,
σύμφωνα με τα οριζόμενα στην ισχύουσα νομοθεσία και στον παρόντα
Κανονισμό. Τα εξεταζόμενα μαθήματα και η ύλη τους, καθώς και
προτεινόμενα συγγράμματα, ορίζονται με απόφαση της Συνέλευσης Τμήματος
και αναρτώνται στην επίσημη ιστοσελίδα του Τμήματος.

Η αίτηση και τα δικαιολογητικά των πτυχιούχων τριτοβάθμιας εκπαίδευσης
υποβάλλονται στη Γραμματεία του Τμήματος από 1 έως 15 Νοεμβρίου κάθε
ακαδημαϊκού έτους, σύμφωνα με τα όσα ορίζονται στην κείμενη νομοθεσία.
Οι κατατακτήριες εξετάσεις διενεργούνται κατά το διάστημα από 1 έως 20
Δεκεμβρίου κάθε ακαδημαϊκού έτους. Ενημέρωση των ενδιαφερομένων γίνεται
από τη Γραμματεία με ανάρτηση σχετικών ανακοινώσεων στην ιστοσελίδα του
Τμήματος.

Η σειρά επιτυχίας των υποψηφίων καθορίζεται από το άθροισμα, της
βαθμολογίας όλων των εξεταζόμενων μαθημάτων. Στη σειρά αυτή
περιλαμβάνονται όσοι έχουν συγκεντρώσει συνολική βαθμολογία τουλάχιστον
τριάντα (30) μονάδες και με την προϋπόθεση ότι έχουν συγκεντρώσει δέκα
(10) μονάδες τουλάχιστον σε καθένα από τα τρία (3) μαθήματα. Η κατάταξη
γίνεται κατά φθίνουσα σειρά βαθμολογίας μέχρι να καλυφθεί το
προβλεπόμενο ποσοστό. Αν υπάρχουν περισσότεροι υποψήφιοι με την ίδια
συνολική βαθμολογία, για την αποφυγή της υπέρβασης λαμβάνεται υπόψη η
κατοχή πτυχίου Τμήματος με συναφή μαθήματα με το Τμήμα κατάταξης, όπως
αυτά ορίζονται από τα αντίστοιχα προγράμματα σπουδών. Αν και ο αριθμός
των συναφών μαθημάτων είναι ίδιος μεταξύ των ισοβαθμούντων υποψηφίων,
γίνεται κλήρωση μεταξύ των ισοδύναμων υποψηφίων. Δεν επιτρέπεται επιλογή
υποψηφίων που ισοβαθμούν με τον τελευταίο κατατασσόμενο στο Τμήμα
υποδοχής ως υπεράριθμων.

Το εξάμηνο κατάταξης πτυχιούχων στο Τμήμα καθορίζεται με απόφαση της
Συνέλευσης Τμήματος και δεν μπορεί να είναι μεγαλύτερο του 5ου εξαμήνου
για τετραετή φοίτηση. Με απόφαση της Συνέλευσης Τμήματος, κατά
περίπτωση, οι κατατασσόμενοι απαλλάσσονται από την εξέταση μαθημάτων του
προγράμματος σπουδών του Τμήματος που διδάχθηκαν πλήρως ή επαρκώς στο
Τμήμα ή τη Σχολή προέλευσης. Με την ίδια απόφαση, οι κατατασσόμενοι
υποχρεώνονται να εξεταστούν σε μαθήματα, τα οποία σύμφωνα με το
πρόγραμμα σπουδών κρίνεται ότι δεν διδάχθηκαν πλήρως ή επαρκώς στο Τμήμα
ή τη Σχολή προέλευσης. Οι κατατασσόμενοι απαλλάσσονται από την εξέταση
των μαθημάτων στα οποία εξετάστηκαν για την κατάταξή τους, εφόσον τα
μαθήματα αυτά αντιστοιχούν σε μαθήματα του Προγράμματος σπουδών του
Τμήματος.

\hypertarget{ux3c0ux3b9ux3c3ux3c4ux3bfux3c0ux3bfux3b9ux3b7ux3c4ux3b9ux3baux3cc-ux3c0ux3b1ux3b9ux3b4ux3b1ux3b3ux3c9ux3b3ux3b9ux3baux3aeux3c2-ux3baux3b1ux3b9-ux3b4ux3b9ux3b4ux3b1ux3baux3c4ux3b9ux3baux3aeux3c2-ux3b5ux3c0ux3acux3c1ux3baux3b5ux3b9ux3b1ux3c2}{%
\subsection{Πιστοποιητικό Παιδαγωγικής και Διδακτικής
Επάρκειας}\label{ux3c0ux3b9ux3c3ux3c4ux3bfux3c0ux3bfux3b9ux3b7ux3c4ux3b9ux3baux3cc-ux3c0ux3b1ux3b9ux3b4ux3b1ux3b3ux3c9ux3b3ux3b9ux3baux3aeux3c2-ux3baux3b1ux3b9-ux3b4ux3b9ux3b4ux3b1ux3baux3c4ux3b9ux3baux3aeux3c2-ux3b5ux3c0ux3acux3c1ux3baux3b5ux3b9ux3b1ux3c2}}

Οι φοιτητές/φοιτήτριες και οι απόφοιτοι/απόφοιτες του Τμήματος έχουν τη
δυνατότητα, αν το θελήσουν, να αποκτήσουν Πιστοποιητικό Παιδαγωγικής και
Διδακτικής Επάρκειας σύμφωνα με τις προϋποθέσεις και τους όρους που
αναγράφονται στον «Κανονισμό Παιδαγωγικής και Διδακτικής Επάρκειας» του
Τμήματος. Τα τέσσερα μαθήματα και οι δύο Πρακτικές Ασκήσεις, στα οποία
πρέπει να επιτύχει όποιος ενδιαφέρεται, ανήκουν στο Πρόγραμμα Σπουδών
και είναι σημειωμένα σε αυτό, με την ένδειξη «Μάθημα για πιστοποιητικό
παιδαγωγικής και διδακτικής επάρκειας». Σημειώνεται ότι κάποια από τα
μαθήματα αυτά, έχουν προαπαιτούμενα μαθήματα. Επίσης σημειώνεται ότι τα
μαθήματα αυτά, καθώς και τις δύο αυτές Πρακτικές Ασκήσεις, μπορεί να τα
δηλώσει και να τα εξεταστεί επιτυχώς οποιοσδήποτε φοιτητής, ακόμα και
εάν δεν ενδιαφέρεται για την απόκτηση Πιστοποιητικού Παιδαγωγικής και
Διδακτικής Επάρκειας και οι μονάδες ECTS τους να μετρήσουν κανονικά για
την απόκτηση πτυχίου.

\hypertarget{ux3c0ux3c1ux3ccux3b3ux3c1ux3b1ux3bcux3bcux3b1-ux3c0ux3c1ux3bfux3c0ux3c4ux3c5ux3c7ux3b9ux3b1ux3baux3ceux3bd-ux3c3ux3c0ux3bfux3c5ux3b4ux3ceux3bd-ux3b1ux3baux3b1ux3b4.-ux3adux3c4ux3bfux3c5ux3c2-2023-24}{%
\section{Πρόγραμμα Προπτυχιακών Σπουδών Ακαδ. Έτους
2023-24}\label{ux3c0ux3c1ux3ccux3b3ux3c1ux3b1ux3bcux3bcux3b1-ux3c0ux3c1ux3bfux3c0ux3c4ux3c5ux3c7ux3b9ux3b1ux3baux3ceux3bd-ux3c3ux3c0ux3bfux3c5ux3b4ux3ceux3bd-ux3b1ux3baux3b1ux3b4.-ux3adux3c4ux3bfux3c5ux3c2-2023-24}}

\hypertarget{ux3c0ux3b5ux3c1ux3b9ux3b5ux3c7ux3ccux3bcux3b5ux3bdux3bf-ux3bcux3b1ux3b8ux3b7ux3bcux3acux3c4ux3c9ux3bd}{%
\section{Περιεχόμενο
Μαθημάτων}\label{ux3c0ux3b5ux3c1ux3b9ux3b5ux3c7ux3ccux3bcux3b5ux3bdux3bf-ux3bcux3b1ux3b8ux3b7ux3bcux3acux3c4ux3c9ux3bd}}

\hypertarget{ux3b5ux3beux3acux3bcux3b7ux3bdux3bf-ux3b1}{%
\subsection{Εξάμηνο Α}\label{ux3b5ux3beux3acux3bcux3b7ux3bdux3bf-ux3b1}}

\hypertarget{ux3b5ux3b9ux3c3ux3b1ux3b3ux3c9ux3b3ux3ae-ux3c3ux3c4ux3b7ux3bd-ux3b5ux3c0ux3b9ux3c3ux3c4ux3aeux3bcux3b7-ux3c4ux3c9ux3bd-ux3c5ux3c0ux3bfux3bbux3bfux3b3ux3b9ux3c3ux3c4ux3ceux3bd-ux3baux3bfux3c1ux3bcux3bfux3cd}{%
\subsubsection{Εισαγωγή στην Επιστήμη των Υπολογιστών
(κορμού)}\label{ux3b5ux3b9ux3c3ux3b1ux3b3ux3c9ux3b3ux3ae-ux3c3ux3c4ux3b7ux3bd-ux3b5ux3c0ux3b9ux3c3ux3c4ux3aeux3bcux3b7-ux3c4ux3c9ux3bd-ux3c5ux3c0ux3bfux3bbux3bfux3b3ux3b9ux3c3ux3c4ux3ceux3bd-ux3baux3bfux3c1ux3bcux3bfux3cd}}

\emph{Η πληροφορική ως επιστήμη. Παρουσίαση της εξελικτικής πορείας της
τεχνολογίας των υπολογιστών. Ο υπολογιστής ως επεξεργαστής δεδομένων. Το
πρόγραμμα επεξεργασίας (λογισμικό). Το υλικό κατά το μοντέλο von
Neumann. Δυαδική αναπαράσταση δεδομένων (bits και bytes, δυαδικοί
αριθμοί, αποθήκευση πληροφορίας κειμένου, εικόνας και ήχου, ακέραιοι
αριθμοί, συμπλήρωμα ως προς 2, αναπαράσταση κινητής υποδιαστολής).
Πράξεις με δυαδικούς αριθμούς (πρόσθεση μη προσημασμένων αριθμών,
πρόσθεση ακεραίων, πράξεις κινητής υποδιαστολής, λογικές πράξεις και
πράξεις ολίσθησης). Οργάνωση υπολογιστών (η κεντρική μονάδα
επεξεργασίας, η κύρια μνήμη και ιεραρχίες μνήμης, διευθυνσιοδότηση,
εκτέλεση εντολών και κύκλος μηχανής, συσκευές και μέθοδοι Εισόδου-Εξόδου
(Ε/Ε), διασύνδεση υποσυστημάτων, δίαυλοι συστήματος). Εισαγωγή στα
Δίκτυα υπολογιστών. Εισαγωγή στα Λειτουργικά Συστήματα. Εισαγωγή στους
αλγορίθμους \& στις Γλώσσες Προγραμματισμού. Εισαγωγή στις Βάσεις
Δεδομένων. Συμπίεση και Ασφάλεια Δεδομένων.}

\hypertarget{ux3b5ux3b9ux3c3ux3b1ux3b3ux3c9ux3b3ux3ae-ux3c3ux3c4ux3bfux3bd-ux3c0ux3c1ux3bfux3b3ux3c1ux3b1ux3bcux3bcux3b1ux3c4ux3b9ux3c3ux3bcux3cc-ux3baux3bfux3c1ux3bcux3bfux3cd}{%
\subsubsection{Εισαγωγή στον Προγραμματισμό
(κορμού)}\label{ux3b5ux3b9ux3c3ux3b1ux3b3ux3c9ux3b3ux3ae-ux3c3ux3c4ux3bfux3bd-ux3c0ux3c1ux3bfux3b3ux3c1ux3b1ux3bcux3bcux3b1ux3c4ux3b9ux3c3ux3bcux3cc-ux3baux3bfux3c1ux3bcux3bfux3cd}}

\emph{Σύντομη εισαγωγή στην πληροφορική και στους υπολογιστές. Η έννοια
του αλγόριθμου ως πεπερασμένη ακολουθία βημάτων για τη λύση προβλημάτων
και των γλωσσών προγραμματισμού ως αυστηρών μέσων έκφρασης αλγορίθμων. Η
γλώσσα ``C'', τα κύρια χαρακτηριστικά της και η διαδικασία μεταγλώττισης
και εκτέλεσης προγραμμάτων. Η δομή του προγράμματος στη γλώσσα ``C'', οι
βασικές προγραμματιστικές εντολές και οι εντολές ελέγχου ροής του
προγράμματος. Απλοί τύποι δεδομένων, ορισμός μεταβλητών, τελεστές και
εκφράσεις. Πίνακες (μονοδιάστατοι και πολυδιάστατοι) και στοιχειώσεις
δομές δεδομένων. Αφηρημένοι τύποι δεδομένων. Αναζήτηση και ταξινόμηση
πινάκων. Απαριθμήσεις, δομές (structures), ενώσεις (unions). Δείκτες
(pointers), σχέση μεταξύ δεικτών και πινάκων, συμβολοσειρών και δεικτών,
μετατροπές τύπων. Δείκτες σε εγγραφές. Δυναμική παραχώρηση μνήμης.
Γραμμικές λίστες, απλά συνδεδεμένες λίστες ουρές, στοίβες, διπλά
συνδεδεμένες λίστες. Δέντρα και γράφοι, δυαδικά δέντρα αναζήτησης.
Εργαστήριο προγραμματισμού (Επιλογή γλώσσας προγραμματισμού: ``C'').}

\hypertarget{ux3bcux3b1ux3b8ux3b7ux3bcux3b1ux3c4ux3b9ux3baux3ccux3c2-ux3bbux3bfux3b3ux3b9ux3c3ux3bcux3ccux3c2-ux3baux3bfux3c1ux3bcux3bfux3cd}{%
\subsubsection{Μαθηματικός Λογισμός
(κορμού)}\label{ux3bcux3b1ux3b8ux3b7ux3bcux3b1ux3c4ux3b9ux3baux3ccux3c2-ux3bbux3bfux3b3ux3b9ux3c3ux3bcux3ccux3c2-ux3baux3bfux3c1ux3bcux3bfux3cd}}

\emph{Βασικά Σύνολα. Πραγματικοί Αριθμοί -- Αξιώματα του R --
Κλειστότητα του R. Μιγαδικοί Αριθμοί. Ευκλείδειοι χώροι. Ακολουθίες.
Μονοτονία -- Φράγματα, Υπακολουθίες, Σύγκλιση. Αριθμητικές Σειρές.
Κριτήρια Σύγκλισης, Απόλυτη και Σχετική Σύγκλιση, Τηλεσκοπικές Σειρές.
Συναρτήσεις μιας μεταβλητής. Πράξεις, Όριο και Συνέχεια, Παράγωγος,
Βασικά Θεωρήματα Διαφορικού Λογισμού, Ακρότατα -- Κυρτότητα, Θεώρημα
Taylor, Σειρές Taylor -- Δυναμοσειρές, Αόριστο Ολοκλήρωμα, Ορισμένο
Ολοκλήρωμα, Γενικευμένα Ολοκληρώματα, Συναρτήσεις Βήτα και Γάμμα,
Εφαρμογές Ολοκληρωμάτων, Διαφορικές εξισώσεις. Συναρτήσεις πολλών
μεταβλητών, Είδη συναρτήσεων, Όριο και Συνέχεια, Κατευθυνόμενη -- Μερική
Παράγωγος, Ακρότατα -- Δεσμευμένα Ακρότατα. Ολοκλήρωση, Διπλή
ολοκλήρωση, Πολλαπλή ολοκλήρωση, Αλλαγή Μεταβλητών, Εφαρμογές πολλαπλής
ολοκλήρωσης, Θεωρία Fourier, FFT.}

\hypertarget{ux3b3ux3c1ux3b1ux3bcux3bcux3b9ux3baux3ae-ux3acux3bbux3b3ux3b5ux3b2ux3c1ux3b1-ux3baux3bfux3c1ux3bcux3bfux3cd}{%
\subsubsection{Γραμμική Άλγεβρα
(κορμού)}\label{ux3b3ux3c1ux3b1ux3bcux3bcux3b9ux3baux3ae-ux3acux3bbux3b3ux3b5ux3b2ux3c1ux3b1-ux3baux3bfux3c1ux3bcux3bfux3cd}}

\emph{Σύνολα. Καρτεσιανά γινόμενα. Σχέσεις. Πράξεις. Αλγεβρικές δομές.
Πίνακες, πράξεις πινάκων, ανάστροφος πίνακας, αντίστροφος πίνακας.
Ορίζουσες και ιδιότητες οριζουσών. Γραμμικά συστήματα. Μέθοδος Gauss.
Μέθοδος Gauss --Jordan. Λύση συστήματος με τον αντίστροφο πίνακα.
Μέθοδος Cramer. Διανυσματικοί χώροι. Γραμμικές απεικονίσεις. Πυρήνας και
εικόνα γραμμικής απεικόνισης. Αλλαγή βάσης. Ιδιοτιμές και
ιδιοδιανύσματα. Διαγωνιοποίηση πίνακα. Εφαρμογές στην πληροφορική.}

\hypertarget{ux3c0ux3bbux3b7ux3c1ux3bfux3c6ux3bfux3c1ux3b9ux3baux3ae-ux3c3ux3c4ux3b9ux3c2-ux3b1ux3bdux3b8ux3c1ux3c9ux3c0ux3b9ux3c3ux3c4ux3b9ux3baux3adux3c2-ux3b5ux3c0ux3b9ux3c3ux3c4ux3aeux3bcux3b5ux3c2-ux3baux3bfux3c1ux3bcux3bfux3cd}{%
\subsubsection{Πληροφορική στις Ανθρωπιστικές Επιστήμες
(κορμού)}\label{ux3c0ux3bbux3b7ux3c1ux3bfux3c6ux3bfux3c1ux3b9ux3baux3ae-ux3c3ux3c4ux3b9ux3c2-ux3b1ux3bdux3b8ux3c1ux3c9ux3c0ux3b9ux3c3ux3c4ux3b9ux3baux3adux3c2-ux3b5ux3c0ux3b9ux3c3ux3c4ux3aeux3bcux3b5ux3c2-ux3baux3bfux3c1ux3bcux3bfux3cd}}

\emph{Κοινωνία της Πληροφορίας. Δεδομένα-Πληροφορία-Γνώση -Σοφία.
Εισαγωγή στο Διαδίκτυο και τον Ιστό. Εικονικά περιβάλλοντα Πληροφόρησης.
Εικονικά Περιβάλλοντα Μάθησης. Μάθηση από Απόσταση. Ηλεκτρονικό
Επιχειρείν. Ηλεκτρονικό εμπόριο. Τηλε-Εργασία. Ηλεκτρονική Διακυβέρνηση.
Ηλεκτρονική Δημοκρατία. Ηλεκτρονική Τραπεζική. Ηλεκτρονική Υγεία. Το
Ψηφιακό Χάσμα.}

\hypertarget{ux3b5ux3beux3acux3bcux3b7ux3bdux3bf-ux3b2}{%
\subsection{Εξάμηνο Β}\label{ux3b5ux3beux3acux3bcux3b7ux3bdux3bf-ux3b2}}

\hypertarget{ux3c0ux3c1ux3bfux3b3ux3c1ux3b1ux3bcux3bcux3b1ux3c4ux3b9ux3c3ux3bcux3ccux3c2-ux3c5ux3c0ux3bfux3bbux3bfux3b3ux3b9ux3c3ux3c4ux3ceux3bd-ux3baux3bfux3c1ux3bcux3bfux3cd}{%
\subsubsection{Προγραμματισμός Υπολογιστών
(κορμού)}\label{ux3c0ux3c1ux3bfux3b3ux3c1ux3b1ux3bcux3bcux3b1ux3c4ux3b9ux3c3ux3bcux3ccux3c2-ux3c5ux3c0ux3bfux3bbux3bfux3b3ux3b9ux3c3ux3c4ux3ceux3bd-ux3baux3bfux3c1ux3bcux3bfux3cd}}

\emph{Τεχνικές για top-down, modular, και δομημένο σχεδιασμό για
παραγωγή προγραμμάτων μεγάλου μεγέθους. Προχωρημένες δυναμικές δομές
δεδομένων. Βασικές τεχνικές επεξεργασίας αρχείων (ακολουθιακές ή τυχαίας
προσπέλασης). Κλάσεις και αντικείμενα. Προγραμματισμός με αντικείμενα.
Τελεστές, μεταβλητές, μέθοδοι, καθοριζόμενοι τελεστές, σχέσεις,
εξαρτήσεις, διαγράμματα κλάσεων. Συναρτήσεις: δήλωση ορισμός υπερφόρτωση
συναρτήσεων. Δείκτες, αναφορές, προχωρημένες συναρτήσεις, υπερφόρτωση
τελεστών. Διατάξεις. Κληρονομικότητα. Πολυμορφισμός. Διαχείριση
εξαιρέσεων, ανίχνευση και χειρισμός λαθών. Προγραμματισμός με πρότυπα
και με βιβλιοθήκες προτύπων. Αντικειμενοστραφής ανάλυση και σχεδίαση.
Σχεδιαστικά υποδείγματα. Προκαθορισμένες βιβλιοθήκες. Εργαστήριο
προγραμματισμού (Επιλογή Γλώσσας: ``C++'').}

\hypertarget{ux3b4ux3bfux3bcux3adux3c2-ux3b4ux3b5ux3b4ux3bfux3bcux3adux3bdux3c9ux3bd-ux3baux3bfux3c1ux3bcux3bfux3cd}{%
\subsubsection{Δομές Δεδομένων
(κορμού)}\label{ux3b4ux3bfux3bcux3adux3c2-ux3b4ux3b5ux3b4ux3bfux3bcux3adux3bdux3c9ux3bd-ux3baux3bfux3c1ux3bcux3bfux3cd}}

\emph{Τύποι και δομές δεδομένων (ορισμοί, χρήσεις, διαχείριση,
εφαρμογές). Στοίβα (stack), βασικές πράξεις, υλοποίηση στοίβας με
πίνακα. Ουρά (queue), βασικές πράξεις, υλοποίηση ουράς με πίνακα. Λίστα
(list), βασικές πράξεις, Συνδεδεμένη λίστα (linked list), υλοποίηση με
χρήση δεικτών, Δένδρα, Δυαδικά Δένδρα (binary trees), βασικές πράξεις,
υλοποίηση ΔΔ με πίνακα, με δείκτες και με αναδρομή. Δένδρα AVL. Δένδρα
Β, βασικές πράξεις. Κατακερματισμός (hashing). Διαχείριση μνήμης.}

\hypertarget{ux3b4ux3b9ux3b1ux3baux3c1ux3b9ux3c4ux3ac-ux3bcux3b1ux3b8ux3b7ux3bcux3b1ux3c4ux3b9ux3baux3ac-ux3baux3bfux3c1ux3bcux3bfux3cd}{%
\subsubsection{Διακριτά Μαθηματικά
(κορμού)}\label{ux3b4ux3b9ux3b1ux3baux3c1ux3b9ux3c4ux3ac-ux3bcux3b1ux3b8ux3b7ux3bcux3b1ux3c4ux3b9ux3baux3ac-ux3baux3bfux3c1ux3bcux3bfux3cd}}

\emph{Εισαγωγή -- αναδρομικά προβλήματα: ο πύργος του Hanoi, διαμέριση
επιπέδου, το πρόβλημα του Flavious Josephus. Βασικές αρχές της
συνδυαστικής ανάλυσης: το αντικείμενο της συνδυαστικής, οι βασικές αρχές
της συνδυαστικής, οι βασικοί συνδυαστικοί σχηματισμοί. Λογισμός με
πεπερασμένα αθροίσματα: ιδιότητες, πολλαπλά αθροίσματα. Διακριτός
λογισμός: αντιστοίχιση διακριτού και απειροστικού λογισμού, αρνητικές
παραγοντικές δυνάμεις, πίνακας διαφορών -- αθροισμάτων. Διωνυμικοί
συντελεστές -- ειδικοί αριθμοί: διωνυμικοί συντελεστές, βασικές
ταυτότητες, αθροίσματα γινομένων, αριθμοί Stirling, βασικές ταυτότητες,
αρμονικοί αριθμοί, αριθμοί Fibonacci, αριθμοί Catalan. Βασικές αρχές
θεωρίας αριθμών: ευκλείδεια διαίρεση, διαιρετότητα, μέγιστος κοινός
διαιρέτης, γραμμική διοφαντική εξίσωση, ελάχιστο κοινό πολλαπλάσιο,
πρώτοι αριθμοί, πλήθος και άθροισμα διαιρετών. Ακέραιες συναρτήσεις --
γεννήτριες συναρτήσεις: ακέραιο μέρος πραγματικού αριθμού, αριθμητικές
-- πολλαπλασιαστικές συναρτήσεις, η συνάρτηση του Euler, η συνάρτηση του
Legendre. Γεννήτρια συνάρτηση: εκθετική γεννήτρια συνάρτηση, γεννήτρια
συνάρτηση αριθμών Catalan, γεννήτρια συνάρτηση αριθμών Fibonacci,
γεννήτρια συνάρτηση αριθμών Stirling, λογισμός με γεννήτριες
συναρτήσεις, πίνακας απλών ακολουθιών και γεννητριών τους, γεννήτριες
συναρτήσεις ειδικών αριθμών.}

\hypertarget{ux3c0ux3b9ux3b8ux3b1ux3bdux3ccux3c4ux3b7ux3c4ux3b5ux3c2-ux3baux3bfux3c1ux3bcux3bfux3cd}{%
\subsubsection{Πιθανότητες
(κορμού)}\label{ux3c0ux3b9ux3b8ux3b1ux3bdux3ccux3c4ux3b7ux3c4ux3b5ux3c2-ux3baux3bfux3c1ux3bcux3bfux3cd}}

\emph{Έννοια πιθανότητας. Αξιωματικός και εμπειρικός ορισμός
πιθανότητας. Χώροι πιθανότητας. Δεσμευμένη πιθανότητα και ανεξαρτησία.
Συνδυαστική ανάλυση. Έννοια τυχαίας μεταβλητής. Μονοδιάστατες κατανομές.
Συναρτήσεις τυχαίας μεταβλητής. Μέση τιμή, ροπές, διασπορά, συντελεστής
συσχέτισης, συναρτήσεις συσχέτισης. Πολυδιάστατες κατανομές. Νόμος του
Bayes. Κεντρικό Οριακό θεώρημα. Ροπογεννήτριες και χαρακτηριστικές
συναρτήσεις. Τυχαίοι περίπατοι. Στοχαστικές διεργασίες. Στάσιμες και
εργοδικές στοχαστικές διεργασίες. Master Equation, Εξίσωση Langevin,
Εξίσωση Fokker-Planck, Αλυσίδες Markov.}

\includegraphics{applied-programming-python.md}

\hypertarget{ux3bfux3c1ux3b3ux3acux3bdux3c9ux3c3ux3b7-ux3baux3b1ux3b9-ux3b4ux3b9ux3bfux3afux3baux3b7ux3c3ux3b7-ux3b5ux3c0ux3b9ux3c7ux3b5ux3b9ux3c1ux3b7ux3c3ux3adux3c9ux3bd-ux3b5ux3c0ux3b9ux3bbux3bfux3b3ux3aeux3c2}{%
\subsubsection{Οργάνωση και Διοίκηση Επιχειρησέων
(επιλογής)}\label{ux3bfux3c1ux3b3ux3acux3bdux3c9ux3c3ux3b7-ux3baux3b1ux3b9-ux3b4ux3b9ux3bfux3afux3baux3b7ux3c3ux3b7-ux3b5ux3c0ux3b9ux3c7ux3b5ux3b9ux3c1ux3b7ux3c3ux3adux3c9ux3bd-ux3b5ux3c0ux3b9ux3bbux3bfux3b3ux3aeux3c2}}

\emph{Οργανωσιακή Θεωρία, Τι είναι Οργανισμός, Διαστάσεις Οργανωσιακού
Σχεδιασμού, Στρατηγική Επιχειρήσεων, Αντιστοίχιση σχεδιασμού-στόxων και
στρατηγικής, Αποτελεσματικότητα έναντι Αποδοτικότητας, Τύποι Δομής
Οργανισμών, Συστήματα Πληροφορικής και Επικοινωνιών για τη λήψη
Επιχειρηματικών Αποφάσεων, Αντίκτυπο της Ποιότητας της Πληροφορίας στη
λήψη αποφάσεων, Επιχειρηματικές Διαδικασίες υποβοηθούμενες από την
τεχνολογία RFID κ.α.}

\hypertarget{ux3c0ux3b1ux3b9ux3b4ux3b1ux3b3ux3c9ux3b3ux3b9ux3baux3ac-ux3b5ux3c0ux3b9ux3bbux3bfux3b3ux3aeux3c2}{%
\subsubsection{Παιδαγωγικά
(επιλογής)}\label{ux3c0ux3b1ux3b9ux3b4ux3b1ux3b3ux3c9ux3b3ux3b9ux3baux3ac-ux3b5ux3c0ux3b9ux3bbux3bfux3b3ux3aeux3c2}}

\emph{Το μάθημα αποσκοπεί στην ανάλυση των θεωριών μάθησης και των αρχών
νευροψυχολογίας, των μοντέλων εκπαίδευσης και των παιδαγωγικών
προσεγγίσεων καθώς και των σύγχρονων κοινωνιολογικών ζητημάτων της
εκπαίδευσης.}

\hypertarget{ux3b5ux3bbux3b5ux3cdux3b8ux3b5ux3c1ux3b7-ux3b5ux3c0ux3b9ux3bbux3bfux3b3ux3ae-ux3b5ux3c0ux3b9ux3bbux3bfux3b3ux3aeux3c2}{%
\subsubsection{Ελεύθερη Επιλογή
(επιλογής)}\label{ux3b5ux3bbux3b5ux3cdux3b8ux3b5ux3c1ux3b7-ux3b5ux3c0ux3b9ux3bbux3bfux3b3ux3ae-ux3b5ux3c0ux3b9ux3bbux3bfux3b3ux3aeux3c2}}

\emph{Μάθημα από άλλο Tμήμα του Ιονίου Πανεπιστημίου. Τα μαθήματα
ελεύθερης επιλογής ΕΕ (δηλαδή μαθήματα επιλογής που διδάσκονται από άλλα
Τμήματα του Ιονίου Πανεπιστημίου) πρέπει α) να αντιστοιχούν σε τρεις (3)
ή περισσότερες διδακτικές μονάδες στο Πρόγραμμα Σπουδών του άλλου
Τμήματος. β) το περιεχόμενό τους να μην είναι Πληροφορική αφού αυτό το
περιεχόμενο καλύπτεται πλήρως στο Τμήμα Πληροφορικής, αλλά να σχετίζεται
με τα αντικείμενα της Πληροφορικής γ) μόνον ένα μάθημα Ελεύθερης
Επιλογής μπορεί να υπολογιστεί στη διαμόρφωση του βαθμού του πτυχίου.
Για να δηλώσει ο φοιτητής μάθημα ελεύθερης επιλογής ΕΕ από άλλο Τμήμα,
θα πρέπει πρώτα να έρθει σε συνεννόηση με τη Γραμματεία του Τμήματος
Πληροφορικής.}

\hypertarget{ux3b5ux3beux3acux3bcux3b7ux3bdux3bf-ux3b3}{%
\subsection{Εξάμηνο Γ}\label{ux3b5ux3beux3acux3bcux3b7ux3bdux3bf-ux3b3}}

\hypertarget{ux3b1ux3c1ux3c7ux3b9ux3c4ux3b5ux3baux3c4ux3bfux3bdux3b9ux3baux3ae-ux3c5ux3c0ux3bfux3bbux3bfux3b3ux3b9ux3c3ux3c4ux3ceux3bd-ux3baux3bfux3c1ux3bcux3bfux3cd}{%
\subsubsection{Αρχιτεκτονική Υπολογιστών
(κορμού)}\label{ux3b1ux3c1ux3c7ux3b9ux3c4ux3b5ux3baux3c4ux3bfux3bdux3b9ux3baux3ae-ux3c5ux3c0ux3bfux3bbux3bfux3b3ux3b9ux3c3ux3c4ux3ceux3bd-ux3baux3bfux3c1ux3bcux3bfux3cd}}

\emph{Εισαγωγή στην αρχιτεκτονική υπολογιστών. Ψηφιακή Λογική:
συνδυαστικά και ακολουθιακά λογικά κυκλώματα. Αρχιτεκτονικές συνόλου
εντολών: τύποι εντολών, κύκλος μηχανής και εκτέλεση εντολών,
αρχιτεκτονικές CISC και RISC. Κεντρική μονάδα επεξεργασίας (ΚΜΕ): δομή
και αρχές λειτουργίας. Απόδοση ΚΜΕ και μετροπρογράμματα. Παραλληλισμός
σε επίπεδο εντολών: ΚΜΕ πολλαπλών κύκλων εκτέλεσης εντολής και
pipelining. Επεξεργαστές superscalar και VLIW. Τεχνολογίες κύριας
μνήμης. Ιεραρχίες μνήμης και κρυφές μνήμες. Εικονική μνήμη, υποστήριξη
από ΚΜΕ. Διασύνδεση Εισόδου-Εξόδου (Ε/Ε), δίαυλοι και ελεγκτές Ε/Ε,
διακοπές και τεχνικές άμεσης προσπέλασης μνήμης (DMA).}

\hypertarget{ux3b1ux3bdux3c4ux3b9ux3baux3b5ux3b9ux3bcux3b5ux3bdux3bfux3c3ux3c4ux3c1ux3b5ux3c6ux3aeux3c2-ux3c0ux3c1ux3bfux3b3ux3c1ux3b1ux3bcux3bcux3b1ux3c4ux3b9ux3c3ux3bcux3ccux3c2-ux3baux3bfux3c1ux3bcux3bfux3cd}{%
\subsubsection{Αντικειμενοστρεφής Προγραμματισμός
(κορμού)}\label{ux3b1ux3bdux3c4ux3b9ux3baux3b5ux3b9ux3bcux3b5ux3bdux3bfux3c3ux3c4ux3c1ux3b5ux3c6ux3aeux3c2-ux3c0ux3c1ux3bfux3b3ux3c1ux3b1ux3bcux3bcux3b1ux3c4ux3b9ux3c3ux3bcux3ccux3c2-ux3baux3bfux3c1ux3bcux3bfux3cd}}

\emph{Εισαγωγή στην έννοια του αντικειμενοστρεφούς προγραμματισμού.
Βασικές έννοιες Java -- Μεταβλητές -- Δεδομένα -- Υπολογισμοί. Δομές
διακλάδωσης, πίνακες. Κλάσεις, Αντικείμενα και Κληρονομικότητα στη Java.
Περιβάλλοντα Αλληλεπίδρασης στη Java. Η έννοια της εξαίρεσης και οι
διάφοροι τρόποι χειρισμού των εξαιρέσεων. Δημιουργία Applets και
χρησιμοποίηση τεχνικών εισόδων -- εξόδων δεδομένων. Νήματα εκτέλεσης
(threads) και παράλληλος προγραμματισμός με τη Java. Java graphics και
animation. Java και προγραμματισμός για το διαδίκτυο. Εργαστήριο
Προγραμματισμού (Επιλογή Γλώσσας: ``Java'').}

\hypertarget{ux3c3ux3c4ux3b1ux3c4ux3b9ux3c3ux3c4ux3b9ux3baux3ae-ux3baux3bfux3c1ux3bcux3bfux3cd}{%
\subsubsection{Στατιστική
(κορμού)}\label{ux3c3ux3c4ux3b1ux3c4ux3b9ux3c3ux3c4ux3b9ux3baux3ae-ux3baux3bfux3c1ux3bcux3bfux3cd}}

\emph{Θεωρία δειγματοληψίας, Τυχαία δείγματα, Τυχαίοι αριθμοί. Περιγραφή
στατιστικών δεδομένων με πίνακες και γραφήματα, Πίνακας συχνοτήτων,
Ραβδόγραμμα, Ιστόγραμμα. Στατιστικές εκτιμήσεις, Διαστήματα
εμπιστοσύνης, Διαδικασία ελέγχου στατιστικής υπόθεσης, Στατιστική
υπόθεση, Στατιστική ελέγχου, Περιοχή απόρριψης, Απόφαση ελέγχου.
Παλινδρόμηση και Συσχέτιση. Η παραβολή ελαχίστων τετραγώνων, Ανάλυση
διασποράς. Συντελεστής συσχέτισης και σημειακή εκτίμηση του. Σχέση
συντελεστή συσχέτισης και παλινδρόμησης. Χρήση στατιστικού λογισμικού.}

\hypertarget{ux3b4ux3b9ux3b4ux3b1ux3baux3c4ux3b9ux3baux3ae-ux3c4ux3b7ux3c2-ux3c0ux3bbux3b7ux3c1ux3bfux3c6ux3bfux3c1ux3b9ux3baux3aeux3c2-ux3baux3bfux3c1ux3bcux3bfux3cd}{%
\subsubsection{Διδακτική της Πληροφορικής
(κορμού)}\label{ux3b4ux3b9ux3b4ux3b1ux3baux3c4ux3b9ux3baux3ae-ux3c4ux3b7ux3c2-ux3c0ux3bbux3b7ux3c1ux3bfux3c6ux3bfux3c1ux3b9ux3baux3aeux3c2-ux3baux3bfux3c1ux3bcux3bfux3cd}}

\emph{Σκοπός του μαθήματος είναι ο προβληματισμός και η απόκτηση
γνώσεων: 1.για θέματα που αφορούν τις σπουδές στην πληροφορική, 2.για
έννοιες που συνδέονται με τις θεωρίες της μάθησης και της διδακτικής της
πληροφορικής. Πιο αναλυτικά στο μάθημα περιλαμβάνονται οι παρακάτω
ενότητες: Η πληροφορική στην εκπαίδευση: γνωστικό αντικείμενο και
εκπαιδευτικό μέσο, Το πρόγραμμα σπουδών πληροφορικής στην ελληνική
εκπαίδευση, ο προγραμματισμός ως γνωστικό αντικείμενο, προγραμματιστικά
εργαλεία. Μάθηση, διδασκαλία και εκπαιδευτικές τεχνικές, μεθοδολογίες
και μέσα διδασκαλίας, σενάρια διδασκαλίας, αξιολόγηση μαθητή}

\hypertarget{ux3bbux3b5ux3b9ux3c4ux3bfux3c5ux3c1ux3b3ux3b9ux3baux3ac-ux3c3ux3c5ux3c3ux3c4ux3aeux3bcux3b1ux3c4ux3b1-ux3baux3bfux3c1ux3bcux3bfux3cd}{%
\subsubsection{Λειτουργικά Συστήματα
(κορμού)}\label{ux3bbux3b5ux3b9ux3c4ux3bfux3c5ux3c1ux3b3ux3b9ux3baux3ac-ux3c3ux3c5ux3c3ux3c4ux3aeux3bcux3b1ux3c4ux3b1-ux3baux3bfux3c1ux3bcux3bfux3cd}}

\emph{Βασικές έννοιες, Δομή ενός Λ.Σ. Διεργασίες: Μοντέλο και υλοποίηση
διεργασιών, Διαδιεργασιακή επικοινωνία, Χρονοπρογραμματισμός διεργασιών.
Συστήματα Διαχείρισης Μνήμης, Εναλλαγή, Κατάτμηση σε σταθερά και
μεταβλητά τμήματα, τεχνικές ελέγχου μεταβολών της μνήμης, Ιδεατή Μνήμη,
Σελιδοποίηση, Αλγόριθμοι Αντικατάστασης Σελίδων, Μοντελοποίηση
Αλγορίθμων. Συστήματα Αρχείων: Αρχεία και Κατάλογοι. Αδιέξοδα: Ανίχνευση
και Επανόρθωση, Αποφυγή, Πρόληψη. Εργαστηριακά, θα ασχοληθούμε με
λειτουργικό σύστημα Unix, βασικές εντολές και προγραμματισμό στο
περιβάλλον του σε όλα τα παραπάνω θέματα.}

\hypertarget{ux3b8ux3b5ux3c9ux3c1ux3afux3b1-ux3c4ux3b7ux3c2-ux3c0ux3bbux3b7ux3c1ux3bfux3c6ux3bfux3c1ux3afux3b1ux3c2-ux3b5ux3c0ux3b9ux3bbux3bfux3b3ux3aeux3c2}{%
\subsubsection{Θεωρία της Πληροφορίας
(επιλογής)}\label{ux3b8ux3b5ux3c9ux3c1ux3afux3b1-ux3c4ux3b7ux3c2-ux3c0ux3bbux3b7ux3c1ux3bfux3c6ux3bfux3c1ux3afux3b1ux3c2-ux3b5ux3c0ux3b9ux3bbux3bfux3b3ux3aeux3c2}}

\emph{Εντροπία, σχετική εντροπία. Ο δεύτερος νόμος της θερμοδυναμικής.
Ιδιότητα «ασυμπτωτικής ισοκατανομής». Εντροπία και στοχαστικές
διαδικασίες. Συμπίεση δεδομένων. Βέλτιστοι Κώδικες, Κώδικας Huffman,
Κώδικας Shannon-Fano-Elias. Αλγοριθμική πολυπλοκότητα Kolmogorov.
Χωρητικότητα καναλιού μετάδοσης. Θεμελιώδες θεώρημα Shannon. Διαφορική
εντροπία. Δίαυλος Gauss. Θεωρία πληροφορίας και προηγμένα θέματα
στατιστικής. Μέγιστη Εντροπία. Κωδικοποίηση πηγής. Προσέγγιση με
διαδικασίες Markov. Σώματα Galois. Συνάρτηση ρυθμού-απώλειας. Σήματα και
θόρυβος. Κώδικες διόρθωσης σφαλμάτων. Κώδικες Hamming, Κώδικες
Reed-Muller. Εφαρμογές Θεωρίας Πληροφορίας στη θεωρία επενδύσεων.}

\hypertarget{ux3baux3c1ux3c5ux3c0ux3c4ux3bfux3b3ux3c1ux3b1ux3c6ux3afux3b1-ux3b5ux3c0ux3b9ux3bbux3bfux3b3ux3aeux3c2}{%
\subsubsection{Κρυπτογραφία
(επιλογής)}\label{ux3baux3c1ux3c5ux3c0ux3c4ux3bfux3b3ux3c1ux3b1ux3c6ux3afux3b1-ux3b5ux3c0ux3b9ux3bbux3bfux3b3ux3aeux3c2}}

\emph{Κλασσικοί Αλγόριθμοι -- Ασφάλεια και Κρυπτανάλυση. Μονοαλφαβητικοί
Αλγόριθμοι Αντικατάστασης: Αλγόριθμος Ολίσθησης, Γενικευμένος Αλγόριθμος
Αντικατάστασης, Αλγόριθμος Affine. Πολυαλφαβητικοί Αλγόριθμοι
Αντικατάστασης: Αλγόριθμος Vigenere, Αλγόριθμος Hill. Κλασσικοί
Αλγόριθμοι Αναδιάταξης: Αλγόριθμος Μετάθεσης. Απόλυτη και Υπολογιστική
Ασφάλεια. Ο Αλγόριθμος One-Time-Pad (OTP). Εντροπία και Ασφάλεια
Κρυπτοαλγορίθμων. Πλεονασμός Φυσικής Γλώσσας και Ασφάλεια. Απόσταση
Ενοποίησης. Τυχαιότητα και Ψευδοτυχαιότητα: Γεννήτορες παραγωγής
ψευδοτυχαιότητας. Μοντέρνα Συμμετρικά Κρυπτοσυστήματα: Αλγόριθμοι
Τμήματος και Αλγόριθμοι Ροής. Αλγόριθμος DES, Αλγόριθμος Triple-DES,
Αλγόριθμος S/DES. Tρόποι λειτουργίας συμμετρικών αλγορίθμων: Τρόποι ECB,
CBC, OFB, CFB, CTR. Ακεραιότητα με Μονόδρομες Συναρτήσεις Hash: Σχεδίαση
και Ασφάλεια συναρτήσεων Hash, εφαρμογές στην ασφάλεια συστημάτων και
δικτύων. Αυθεντικότητα με Συναρτήσεις MAC: Σχεδίαση και ασφάλεια
συναρτήσεων MAC, εφαρμογές στην ασφάλεια συστημάτων και δικτύων.
Συνδυασμένες υπηρεσίες: Εμπιστευτικότητα και Αυθεντικότητα με συμμετρικά
συστήματα. Ασύμμετρα Συστήματα ΔΚ. Κρυπτογράφηση με τον Αλγόριθμο RSA. O
Αλγόριθμος Rabin. Ντετερμινιστική και Πιθανοτική Κρυπτογράφηση με ΔΚ. Ο
Αλγόριθμος κρυπτογράφησης Elgamal. Ο Αλγόριθμος Κρυπτογράφησης
Goldwasser-Micali. Ψηφιακή Υπογραφή με αλγορίθμους ΔΚ. Ψηφιακή Υπογραφή
με τον Αλγόριθμο RSA. Συνδυασμένες υπηρεσίες: Εμπιστευτικότητα και
Αυθεντικότητα με κρυπτοσυστήματα ΔΚ. Διαχείριση Δημόσιου Κλειδιού:
Πιστοποιητικά Χ.509. Κεντρικά Μοντέλα Εμπιστοσύνης -- Υποδομές ΔΚ:
Ιεραρχική πιστοποίηση, Δια-πιστοποίηση, Ιεραρχίες Πολλών Επιπέδων.
Μοντέλα Κατανεμημένης Εμπιστοσύνης. Το μοντέλο PGP. Εφαρμογές
κρυπτοαλγορίθμων στην ασφάλεια συστημάτων και δικτύων.}

\hypertarget{ux3b5ux3c0ux3b9ux3baux3bfux3b9ux3bdux3c9ux3bdux3afux3b1-ux3b1ux3bdux3b8ux3c1ux3ceux3c0ux3bfux3c5-ux3c5ux3c0ux3bfux3bbux3bfux3b3ux3b9ux3c3ux3c4ux3ae-ux3b5ux3c0ux3b9ux3bbux3bfux3b3ux3aeux3c2}{%
\subsubsection{Επικοινωνία Ανθρώπου-Υπολογιστή
(επιλογής)}\label{ux3b5ux3c0ux3b9ux3baux3bfux3b9ux3bdux3c9ux3bdux3afux3b1-ux3b1ux3bdux3b8ux3c1ux3ceux3c0ux3bfux3c5-ux3c5ux3c0ux3bfux3bbux3bfux3b3ux3b9ux3c3ux3c4ux3ae-ux3b5ux3c0ux3b9ux3bbux3bfux3b3ux3aeux3c2}}

\emph{Ανάγκες χρήστη στο σχεδιασμό των user-interfaces. Ανθρωποκεντρική
αποτίμηση και στρατηγικές σχεδιασμού και δοκιμής των user interfaces,
τεχνικές αλληλεπίδρασης. Σχεδίαση, προγραμματισμός και πρότυπη διεπαφή.
Τεχνικές και μέθοδοι αξιολόγησης των αποτελεσμάτων ως προς την
ευχρηστία.}

\hypertarget{ux3b5ux3beux3acux3bcux3b7ux3bdux3bf-ux3b4}{%
\subsection{Εξάμηνο Δ}\label{ux3b5ux3beux3acux3bcux3b7ux3bdux3bf-ux3b4}}

\hypertarget{ux3b1ux3bbux3b3ux3ccux3c1ux3b9ux3b8ux3bcux3bfux3b9-ux3baux3bfux3c1ux3bcux3bfux3cd}{%
\subsubsection{Αλγόριθμοι
(κορμού)}\label{ux3b1ux3bbux3b3ux3ccux3c1ux3b9ux3b8ux3bcux3bfux3b9-ux3baux3bfux3c1ux3bcux3bfux3cd}}

\emph{Η έννοια του αλγορίθμου και της πολυπλοκότητας. Βασικές έννοιες
της ανάλυσης αλγορίθμων. Μαθηματικό υπόβαθρο. Τεχνικές επίλυσης
αναδρομικών εξισώσεων. Τεχνικές σχεδίασης αλγορίθμων. Η τεχνική «διαίρει
και βασίλευε». Ο αλγόριθμος της συγχώνευσης. Ο αλγόριθμος της γρήγορης
ταξινόμησης. Ελάχιστος χρόνος εκτέλεσης αλγορίθμων διάταξης.
Πολλαπλασιασμός αριθμών και πινάκων. Η τεχνική του δυναμικού
προγραμματισμού. Ιδιότητα βέλτιστων επιμέρους δομών. Το πρόβλημα του
πολλαπλασιασμού ακολουθίας πινάκων. Το ακέραιο πρόβλημα του σακιδίου. Το
πρόβλημα της διαμέρισης. Η άπληστη τεχνική. Δρομολόγηση εργασιών,
απληστία και ρέστα, το κλασματικό πρόβλημα του σακιδίου. Θεωρία
Γραφημάτων. Αναπαράσταση γραφημάτων, αλγόριθμοι εξερεύνησης γραφημάτων.
Αναζήτηση πρώτα σε πλάτος, αναζήτηση πρώτα σε βάθος. Τοπολογική
ταξινόμηση. Ελάχιστα επικαλύπτοντα δένδρα. Άπληστος υπολογισμός
ελάχιστου επικαλύπτοντος δέντρου. Συντομότερα μονοπάτια. Συντομότερα
μονοπάτια μοναδικής πηγής. Συντομότερα μονοπάτια για όλα τα ζεύγη
κορυφών. Οπισθοδρόμηση. Διακλάδωση και Φράξιμο. Βασικοί αλγόριθμοι
συμβολοσειρών. Εισαγωγή στη Θεωρία Υπολογιστικής Πολυπλοκότητας.}

\hypertarget{ux3b2ux3acux3c3ux3b5ux3b9ux3c2-ux3b4ux3b5ux3b4ux3bfux3bcux3adux3bdux3c9ux3bd-ux3b9-ux3baux3bfux3c1ux3bcux3bfux3cd}{%
\subsubsection{Βάσεις Δεδομένων Ι
(κορμού)}\label{ux3b2ux3acux3c3ux3b5ux3b9ux3c2-ux3b4ux3b5ux3b4ux3bfux3bcux3adux3bdux3c9ux3bd-ux3b9-ux3baux3bfux3c1ux3bcux3bfux3cd}}

\emph{Εισαγωγή στα συστήματα διαχείρισης βάσεων δεδομένων. Φυσική
αποθήκευση στο δίσκο. Μοντελοποίηση σχεσιακών βάσεων δεδομένων (μοντέλο
Οντοτήτων-Συσχετίσεων -- ER, σχεσιακό μοντέλο). Σχεσιακή άλγεβρα. Αρχές
Κανονικοποίησης, γλώσσες επερωτήσεων (η γλώσσα SQL) και συστήματα
τέταρτης γενιάς (4GLs), Πίνακες -- Δημιουργία πινάκων και συσχετίσεων --
Ερωτήσεις (απλές, αριθμητικές) με χρήση της QBE (MS-Access) και της SQL.
Θέματα Ασφάλειας.}

\hypertarget{ux3b4ux3afux3baux3c4ux3c5ux3b1-i-ux3baux3bfux3c1ux3bcux3bfux3cd}{%
\subsubsection{Δίκτυα I
(κορμού)}\label{ux3b4ux3afux3baux3c4ux3c5ux3b1-i-ux3baux3bfux3c1ux3bcux3bfux3cd}}

\emph{Τηλεπικοινωνίες και Δίκτυα. Σήματα (αναλογικά-ψηφιακά), αρχές
μετάδοσης δεδομένων, κωδικοποίηση δεδομένων. Μέσα μετάδοσης: Καλώδια
συνεστραμμένου ζεύγους, ομοαξονικά καλώδια, οπτικές ίνες. Πρότυπα
ενσύρματων δικτύων. Ασύρματα δίκτυα επικοινωνίας με ραδιοκύματα,
μικροκύματα, υπέρυθρες. Πρότυπα ασύρματων δικτύων. Αρχιτεκτονικές
πρωτοκόλλων: Το πρότυπο OSI, το πρότυπο TCP/IP. Τοπικά δίκτυα.
Μητροπολιτικά δίκτυα -- Δίκτυα Ευρείας Περιοχής. Διασύνδεση δικτύων και
δικτυακές συσκευές. Μεταφορά και Δρομολόγηση πακέτων. Τεχνικές
μεταγωγής, Δίκτυα κορμού, αστική και εταιρική πρόσβαση στο Διαδίκτυο.
Πρόσβαση PSTN, ISDN. Τεχνολογίες ευρυζωνικής πρόσβασης (DSL, Wi-fi,
Wi-USB, Wi-Max), υπηρεσίες τρίτης γενιάς (3G). Διαχείριση Δικτύων.}

\hypertarget{ux3b5ux3b9ux3c3ux3b1ux3b3ux3c9ux3b3ux3ae-ux3c3ux3c4ux3b1-ux3c0ux3bbux3b7ux3c1ux3bfux3c6ux3bfux3c1ux3b9ux3b1ux3baux3ac-ux3c3ux3c5ux3c3ux3c4ux3aeux3bcux3b1ux3c4ux3b1-ux3baux3bfux3c1ux3bcux3bfux3cd}{%
\subsubsection{Εισαγωγή στα Πληροφοριακά Συστήματα
(κορμού)}\label{ux3b5ux3b9ux3c3ux3b1ux3b3ux3c9ux3b3ux3ae-ux3c3ux3c4ux3b1-ux3c0ux3bbux3b7ux3c1ux3bfux3c6ux3bfux3c1ux3b9ux3b1ux3baux3ac-ux3c3ux3c5ux3c3ux3c4ux3aeux3bcux3b1ux3c4ux3b1-ux3baux3bfux3c1ux3bcux3bfux3cd}}

\emph{Η έννοια του συστήματος. Νόμοι και αρχές της Γενικής Θεωρίας
Συστημάτων (δομή, όρια, εντροπία, κ.ά.). Μεθοδολογίες Δύσκαμπτων κι
Ευμετάβλητων Συστημάτων. Θεωρία Ευμετάβλητων Συστημάτων του P.
Checkland. Ο στρατηγικός ρόλος των Π.Σ. Κύκλος ζωής πληροφοριακών
συστημάτων. Τεχνικές περιγραφής και ανάλυσης της δομής ενός
πληροφοριακού συστήματος. Στρατηγικές και Μεθοδολογίες ανάπτυξης
πληροφοριακών συστημάτων. Ποιότητα και παράγοντες επιτυχίας ενός Π.Σ.
Οργανισμοί και λειτουργικές διαδικασίες. Πληροφορία, μάνατζμεντ και λήψη
αποφάσεων. Οργανωτικός ανασχεδιασμός και ανασχεδιασμός επιχειρησιακών
διαδικασιών.}

\hypertarget{ux3b1ux3c3ux3c6ux3acux3bbux3b5ux3b9ux3b1-ux3c5ux3c0ux3bfux3bbux3bfux3b3ux3b9ux3c3ux3c4ux3ceux3bd-ux3baux3b1ux3b9-ux3c0ux3c1ux3bfux3c3ux3c4ux3b1ux3c3ux3afux3b1-ux3b4ux3b5ux3b4ux3bfux3bcux3adux3bdux3c9ux3bd-ux3baux3bfux3c1ux3bcux3bfux3cd}{%
\subsubsection{Ασφάλεια υπολογιστών και προστασία δεδομένων
(κορμού)}\label{ux3b1ux3c3ux3c6ux3acux3bbux3b5ux3b9ux3b1-ux3c5ux3c0ux3bfux3bbux3bfux3b3ux3b9ux3c3ux3c4ux3ceux3bd-ux3baux3b1ux3b9-ux3c0ux3c1ux3bfux3c3ux3c4ux3b1ux3c3ux3afux3b1-ux3b4ux3b5ux3b4ux3bfux3bcux3adux3bdux3c9ux3bd-ux3baux3bfux3c1ux3bcux3bfux3cd}}

\emph{Εισαγωγή στην Ασφάλεια: Βασικοί ορισμοί, Μοντέλο Απειλών,
Υπηρεσίες Ασφάλειας. Εισαγωγή στις έννοιες: Απειλή, Ευπάθεια, Κίνδυνος.
Ασφάλεια Συστήματος -- Έλεγχος Λογικής Πρόσβασης -- Τοπική και
Απομακρυσμένη Αυθεντικοποίηση Οντότητας: Κωδικοί Passwords,
Απομακρυσμένη Αυθεντικοποίηση με κρυπτογραφικές τεχνικές, Κωδικοί μιας
χρήσης, Ταυτοποίηση με Μηδενική Γνώση. Έλεγχος Λογικής Πρόσβασης --
Εξουσιοδότηση: Πολιτικές και μοντέλα εξουσιοδότησης (MAC, DAC, RBAC).
Ασφάλεια Λειτουργικού Συστήματος. Κακόβουλο λογισμικό: Μοντέλο Απειλών,
μηχανισμοί αντιμετώπισης, ερευνητικά θέματα. Αυθεντικοποιημένη εδραίωση
κλειδιού και Εφαρμογές: Συστήματα Διανομής κλειδιού, Συστήματα Μεταφοράς
Κλειδιού, Συστήματα Συμφωνίας Κλειδιού. Ασφάλεια Δικτύων: Μοντέλο
απειλών στο Επίπεδο TCP/IP, ασφάλεια υπηρεσιών Διαδικτύου, ασφάλεια στο
Web. Δικτυακά Συστήματα Firewalls.}

\hypertarget{ux3b8ux3b5ux3c9ux3c1ux3afux3b1-ux3c5ux3c0ux3bbux3bfux3b3ux3b9ux3c3ux3bcux3bfux3cd-ux3b5ux3c0ux3b9ux3bbux3bfux3b3ux3aeux3c2}{%
\subsubsection{Θεωρία Υπλογισμού
(επιλογής)}\label{ux3b8ux3b5ux3c9ux3c1ux3afux3b1-ux3c5ux3c0ux3bbux3bfux3b3ux3b9ux3c3ux3bcux3bfux3cd-ux3b5ux3c0ux3b9ux3bbux3bfux3b3ux3aeux3c2}}

\emph{Αλφάβητα και γλώσσες. Πεπερασμένα αυτόματα. Ιδιότητες των
πεπερασμένων αυτομάτων και των γλωσσών που δέχονται. Κανονικές εκφράσεις
και κανονικές γλώσσες. Ισοδυναμία πεπερασμένων αυτομάτων και κανονικών
εκφράσεων. Λήμμα άντλησης για κανονικές γλώσσες. Γραμματικές και η
ιεραρχία του Chomsky. Γραμματικές και γλώσσες χωρίς συμφραζόμενα.
Αυτόματα στοίβας και λήμμα άντλησης για γλώσσες χωρίς συμφραζόμενα.
Ισοδυναμία γραμματικών χωρίς συμφραζόμενα και αυτομάτων στοίβας. Η
έννοια της υπολογισιμότητας. Mηχανές Turing. Aποφασίσιμες και
απαριθμήσιμες γλώσσες. Η θέση των Church-Turing. Eπιλύσιμα και μη
επιλύσιμα προβλήματα. Το πρόβλημα του τερματισμού (halting problem).
Εισαγωγή στην υπολογιστική πολυπλοκότητα. Χρονική πολυπλοκότητα, η κλάση
Ρ, η θέση των Cook-Karp. Αναγωγή και πληρότητα. Μη-ντετερμινισμός και
NP-πληρότητα, σχέση Ρ και ΝΡ, αλγοριθμικές συνέπειες NP-πληρότητας.
Πολυπλοκότητα χώρου, η κλάση PSPACE, το θεώρημα του Savitch.
PSPACE-πλήρη προβλήματα.}

\hypertarget{ux3baux3b9ux3bdux3b7ux3c4ux3ac-ux3baux3b1ux3b9-ux3baux3bfux3b9ux3bdux3c9ux3bdux3b9ux3baux3ac-ux3bcux3adux3c3ux3b1-ux3b5ux3c0ux3b9ux3bbux3bfux3b3ux3aeux3c2}{%
\subsubsection{Κινητά και Κοινωνικά Μέσα
(επιλογής)}\label{ux3baux3b9ux3bdux3b7ux3c4ux3ac-ux3baux3b1ux3b9-ux3baux3bfux3b9ux3bdux3c9ux3bdux3b9ux3baux3ac-ux3bcux3adux3c3ux3b1-ux3b5ux3c0ux3b9ux3bbux3bfux3b3ux3aeux3c2}}

\emph{Συνεργατικός Υπολογισμός. Συστήματα CSCW (Computer-Supported
Cooperative Work). Ταξινόμηση Χώρου-Χρόνου. Ανάπτυξη εφαρμογών.
Εφαρμογές σε Μέσα Κοινωνικής Δικτύωσης. Κινητός Υπολογισμός. Κινητές
Εφαρμογές.}

\hypertarget{ux3b5ux3beux3acux3bcux3b7ux3bdux3bf-ux3b5}{%
\subsection{Εξάμηνο Ε}\label{ux3b5ux3beux3acux3bcux3b7ux3bdux3bf-ux3b5}}

\hypertarget{ux3b4ux3afux3baux3c4ux3c5ux3b1-ii-ux3baux3bfux3c1ux3bcux3bfux3cd}{%
\subsubsection{Δίκτυα II
(κορμού)}\label{ux3b4ux3afux3baux3c4ux3c5ux3b1-ii-ux3baux3bfux3c1ux3bcux3bfux3cd}}

\emph{Σχεδιασμός και ανάπτυξη δικτύων υψηλών ταχυτήτων. Φυσικά Μέσα
Μετάδοσης δικτύων υψηλών ταχυτήτων. Δίκτυα Frame Relay. Δίκτυα ΑΤΜ.
Δίκτυα μεταγωγής Νοητού Κυκλώματος. Ασύρματα δίκτυα υψηλών ταχυτήτων.
Σύγκλιση Τεχνολογιών και Δικτύων. Διαχείριση δικτύων TCP/IP. Πρωτόκολλο
SNMP. Βάση Πληροφορίας Διαχείρισης. Διαχείριση δικτύων OSI. Πρωτόκολλο
CMIP. Δένδρο Πληροφορίας Διαχείρισης. Διαχείριση γεφυρωμένων δικτύων.
Σύγχρονες τεχνικές/μεθοδολογίες διαχείρισης WBM, CORBA, Java-based.}

\hypertarget{ux3b2ux3acux3c3ux3b5ux3b9ux3c2-ux3b4ux3b5ux3b4ux3bfux3bcux3adux3bdux3c9ux3bd-ux3b9ux3b9-ux3baux3bfux3c1ux3bcux3bfux3cd}{%
\subsubsection{Βάσεις Δεδομένων ΙΙ
(κορμού)}\label{ux3b2ux3acux3c3ux3b5ux3b9ux3c2-ux3b4ux3b5ux3b4ux3bfux3bcux3adux3bdux3c9ux3bd-ux3b9ux3b9-ux3baux3bfux3c1ux3bcux3bfux3cd}}

\emph{Προχωρημένα Θέματα SQL (PL SQL). Μοντελοποίηση αντικειμενοστρεφών
και αντικειμενο-σχεσιακών βάσεων δεδομένων, μοντελοποίηση ημι-δομημένης
πληροφορίας (η γλώσσα XML). Οργάνωση Αρχείων και Ευρετήρια (B-trees, B+
trees, Hashing, BitMap). Επεξεργασία και Βελτιστοποίηση Ερωτήσεων.
Διαχείριση συναλλαγών (συγχρονισμός -- ταυτοχρονισμός). Παράλληλες --
Κατανεμημένες βάσεις δεδομένων (αρχιτεκτονική client-server, διασπορά --
αντιγραφή -- τοποθέτηση δεδομένων, μη παραδοσιακές βάσεις δεδομένων
(χωρικές, χωροχρονικές, πολυμέσων), εισαγωγή στις αποθήκες δεδομένων και
την εξόρυξη γνώσης από μεγάλες βάσεις δεδομένων.}

\hypertarget{ux3bbux3b5ux3b9ux3c4ux3bfux3c5ux3c1ux3b3ux3b9ux3baux3ac-ux3c3ux3c5ux3c3ux3c4ux3aeux3bcux3b1ux3c4ux3b1-ux3baux3bfux3c1ux3bcux3bfux3cd-1}{%
\subsubsection{Λειτουργικά Συστήματα
(κορμού)}\label{ux3bbux3b5ux3b9ux3c4ux3bfux3c5ux3c1ux3b3ux3b9ux3baux3ac-ux3c3ux3c5ux3c3ux3c4ux3aeux3bcux3b1ux3c4ux3b1-ux3baux3bfux3c1ux3bcux3bfux3cd-1}}

\emph{Βασικές έννοιες, Δομή ενός Λ.Σ. Διεργασίες: Μοντέλο και υλοποίηση
διεργασιών, Διαδιεργασιακή επικοινωνία, Χρονοπρογραμματισμός διεργασιών.
Συστήματα Διαχείρισης Μνήμης, Εναλλαγή, Κατάτμηση σε σταθερά και
μεταβλητά τμήματα, τεχνικές ελέγχου μεταβολών της μνήμης, Ιδεατή Μνήμη,
Σελιδοποίηση, Αλγόριθμοι Αντικατάστασης Σελίδων, Μοντελοποίηση
Αλγορίθμων. Συστήματα Αρχείων: Αρχεία και Κατάλογοι. Αδιέξοδα: Ανίχνευση
και Επανόρθωση, Αποφυγή, Πρόληψη. Εργαστηριακά, θα ασχοληθούμε με
λειτουργικό σύστημα Unix, βασικές εντολές και προγραμματισμό στο
περιβάλλον του σε όλα τα παραπάνω θέματα.}

\hypertarget{ux3b1ux3bdux3acux3bbux3c5ux3c3ux3b7-ux3baux3b1ux3b9-ux3c3ux3c7ux3b5ux3b4ux3b9ux3b1ux3c3ux3bcux3ccux3c2-ux3c0.ux3c3.-ux3baux3b1ux3c4ux3b5ux3cdux3b8ux3c5ux3bdux3c3ux3b7ux3c2-ux3c0ux3c3}{%
\subsubsection{Ανάλυση και Σχεδιασμός Π.Σ. (κατεύθυνσης
ΠΣ)}\label{ux3b1ux3bdux3acux3bbux3c5ux3c3ux3b7-ux3baux3b1ux3b9-ux3c3ux3c7ux3b5ux3b4ux3b9ux3b1ux3c3ux3bcux3ccux3c2-ux3c0.ux3c3.-ux3baux3b1ux3c4ux3b5ux3cdux3b8ux3c5ux3bdux3c3ux3b7ux3c2-ux3c0ux3c3}}

\emph{Μοντέλα διεργασίας ανάπτυξης Π.Σ. Αναλυτική περιγραφή του κύκλου
ζωής ενός Π.Σ. (καθορισμός προβλήματος, μελέτη σκοπιμότητας, ανάλυση
απαιτήσεων, λογικός και φυσικός σχεδιασμός, εγκατάσταση και συντήρηση).
Μέθοδοι και τεχνικές συλλογής και ανάλυσης απαιτήσεων χρηστών. Ο ρόλος
του αναλυτή. Αντικειμενοστρεφής σχεδίαση συστημάτων. Ανάλυση και
σχεδίαση ΠΣ με χρήση της γλώσσας UML.}

\hypertarget{ux3c0ux3bfux3bbux3b9ux3c4ux3b9ux3baux3adux3c2-ux3baux3b1ux3b9-ux3c4ux3b5ux3c7ux3bdux3bfux3bbux3bfux3b3ux3afux3b5ux3c2-ux3c0ux3c1ux3bfux3c3ux3c4ux3b1ux3c3ux3afux3b1ux3c2-ux3c4ux3b7ux3c2-ux3b9ux3b4ux3b9ux3c9ux3c4ux3b9ux3baux3ccux3c4ux3b7ux3c4ux3b1ux3c2-ux3baux3b1ux3c4ux3b5ux3cdux3b8ux3c5ux3bdux3c3ux3b7ux3c2-ux3c0ux3c3}{%
\subsubsection{Πολιτικές και Τεχνολογίες Προστασίας της Ιδιωτικότητας
(κατεύθυνσης
ΠΣ)}\label{ux3c0ux3bfux3bbux3b9ux3c4ux3b9ux3baux3adux3c2-ux3baux3b1ux3b9-ux3c4ux3b5ux3c7ux3bdux3bfux3bbux3bfux3b3ux3afux3b5ux3c2-ux3c0ux3c1ux3bfux3c3ux3c4ux3b1ux3c3ux3afux3b1ux3c2-ux3c4ux3b7ux3c2-ux3b9ux3b4ux3b9ux3c9ux3c4ux3b9ux3baux3ccux3c4ux3b7ux3c4ux3b1ux3c2-ux3baux3b1ux3c4ux3b5ux3cdux3b8ux3c5ux3bdux3c3ux3b7ux3c2-ux3c0ux3c3}}

\emph{Η έννοια της ιδιωτικότητας, μορφές και τύποι ιδιωτικότητας.
Εννοιολογικό πλαίσιο πληροφοριακής ιδιωτικότητας. Οι ιδιότητες της
ανωνυμίας, μη παρατηρησιμότητας, μη ανιχνευσιμότητας, μη συνδεσιμότητας.
Απειλές κατά της Πληροφοριακής Ιδιωτικότητας. Γενικός Κανονισμός για την
Προστασία Δεδομένων. Πολιτικές ιδιωτικότητας. Δομή, περιεχόμενα,
παρουσίαση και ανάλυση μελετών περίπτωσης. Ανάλυση επικινδυνότητας
πληροφοριακής ιδιωτικότητας. Τεχνολογίες ενίσχυσης της ιδιωτικότητας.
Εργαλεία λογισμικού για την ανωνυμία (anonymizers), πλαίσια πιστοποίησης
(TRUSTe), ανάλυση προτιμήσεων χρήστη (P3P), εργαλεία λογισμικού για την
παραγωγή ψευδο-ταυτότητας (LPWA), onion routing. Ενσωμάτωση απαιτήσεων
ασφάλειας από το σχεδιασμό συστημάτων. Εννοιολογικό πλαίσιο ασφάλειας
πληροφοριών. Διάκριση ιδιοτήτων ασφάλειας πληροφοριών από τις ιδιότητες
της πληροφοριακής ιδιωτικότητας και αναγνώριση κοινών στοιχείων. Ανάλυση
και διαχείριση επικινδυνότητας ασφάλειας πληροφοριών. Ανάλυση αντικτύπου
για την προστασία δεδομένων. Μέθοδοι και εργαλεία λογισμικού για την
υλοποίηση ανάλυσης αντικτύπου παραβίασης προσωπικών δεδομένων.Η έννοια
της ιδιωτικότητας, μορφές και τύποι ιδιωτικότητας. Εννοιολογικό πλαίσιο
πληροφοριακής ιδιωτικότητας. Οι ιδιότητες της ανωνυμίας, μη
παρατηρησιμότητας, μη ανιχνευσιμότητας, μη συνδεσιμότητας. Απειλές κατά
της Πληροφοριακής Ιδιωτικότητας. Γενικός Κανονισμός για την Προστασία
Δεδομένων. Πολιτικές ιδιωτικότητας. Δομή, περιεχόμενα, παρουσίαση και
ανάλυση μελετών περίπτωσης. Ανάλυση επικινδυνότητας πληροφοριακής
ιδιωτικότητας. Τεχνολογίες ενίσχυσης της ιδιωτικότητας. Εργαλεία
λογισμικού για την ανωνυμία (anonymizers), πλαίσια πιστοποίησης
(TRUSTe), ανάλυση προτιμήσεων χρήστη (P3P), εργαλεία λογισμικού για την
παραγωγή ψευδο-ταυτότητας (LPWA), onion routing. Ενσωμάτωση απαιτήσεων
ασφάλειας από το σχεδιασμό συστημάτων. Εννοιολογικό πλαίσιο ασφάλειας
πληροφοριών. Διάκριση ιδιοτήτων ασφάλειας πληροφοριών από τις ιδιότητες
της πληροφοριακής ιδιωτικότητας και αναγνώριση κοινών στοιχείων. Ανάλυση
και διαχείριση επικινδυνότητας ασφάλειας πληροφοριών. Ανάλυση αντικτύπου
για την προστασία δεδομένων. Μέθοδοι και εργαλεία λογισμικού για την
υλοποίηση ανάλυσης αντικτύπου παραβίασης προσωπικών δεδομένων.}

\hypertarget{ux3c0ux3bfux3bbux3c5ux3bcux3adux3c3ux3b1-ux3baux3b1ux3c4ux3b5ux3cdux3b8ux3c5ux3bdux3c3ux3b7ux3c2-ux3c0ux3b1ux3baux3b5}{%
\subsubsection{Πολυμέσα (κατεύθυνσης
ΠΑΚΕ)}\label{ux3c0ux3bfux3bbux3c5ux3bcux3adux3c3ux3b1-ux3baux3b1ux3c4ux3b5ux3cdux3b8ux3c5ux3bdux3c3ux3b7ux3c2-ux3c0ux3b1ux3baux3b5}}

\emph{Υπερμέσα. Συμμετοχικά πολυμέσα. Εικονική πραγματικότητα. Αφήγηση.
Διάδραση. Μορφές αναπαράστασης πληροφορίας σε συστήματα πολυμέσων. Η
αρχιτεκτονική συστημάτων υπερμέσων. Ψυχαγωγικές και Εκπαιδευτικές
Εφαρμογές. Γεωγραφικά Συστήματα Πληροφόρησης. Προγραμματισμός και
ανάπτυξη πολυμεσικών εφαρμογών.Αντικειμενικοί Στόχοι Επιδιωκόμενα
Μαθησιακά Αποτελέσματα. Με την επιτυχή ολοκλήρωση του μαθήματος ο
φοιτητής / τρια θα είναι σε θέση να: Αναλύει τα δομικά στοιχεία
πολυμεσικών εφαρμογών Κάνει σχεδίαση και κατασκευή πολυμεσικών
εφαρμογών.}

\hypertarget{ux3b3ux3c1ux3b1ux3c6ux3b9ux3baux3ac-ux3bcux3b5-ux3c5ux3c0ux3bfux3bbux3bfux3b3ux3b9ux3c3ux3c4ux3adux3c2-ux3baux3b1ux3c4ux3b5ux3cdux3b8ux3c5ux3bdux3c3ux3b7ux3c2-ux3c0ux3b1ux3baux3b5}{%
\subsubsection{Γραφικά με Υπολογιστές (κατεύθυνσης
ΠΑΚΕ)}\label{ux3b3ux3c1ux3b1ux3c6ux3b9ux3baux3ac-ux3bcux3b5-ux3c5ux3c0ux3bfux3bbux3bfux3b3ux3b9ux3c3ux3c4ux3adux3c2-ux3baux3b1ux3c4ux3b5ux3cdux3b8ux3c5ux3bdux3c3ux3b7ux3c2-ux3c0ux3b1ux3baux3b5}}

\emph{Βασικές έννοιες γραφικών με υπολογιστές και εφαρμογές τους,
Ιστορία και γενικά χαρακτηριστικά, Διανυσματική(ά) / Πλεγματική(ά)
απεικόνιση/γραφικά, Εισαγωγικά στοιχεία σωλήνωσης, Βασικές Έννοιες
Σχεδίασης, Αλγόριθμοι Σχεδίασης Ευθύγραμμου Τμήματος, Κύκλου, Έλλειψης,
Φαινόμενο Ταύτισης, Τρόποι αναπαράστασης, παραγωγής και απεικόνισης
τριδιάστατων δεδομένων, Χρωματισμός πολυγώνων, Αποκοπή, Μετασχηματισμοί
και συστήματα συντεταγμένων, Σύνθεση 2Δ Μετασχηματισμών, Ομογενείς
Συντεταγμένες, Μετασχηματισμοί Προβολής, Προοπτική, Παράλληλη, Πλάγια
Προβολή}

\hypertarget{ux3c8ux3b7ux3c6ux3b9ux3b1ux3baux3ae-ux3b5ux3c0ux3b5ux3beux3b5ux3c1ux3b3ux3b1ux3c3ux3afux3b1-ux3b5ux3b9ux3baux3ccux3bdux3b1ux3c2-ux3b5ux3c0ux3b9ux3bbux3bfux3b3ux3aeux3c2}{%
\subsubsection{Ψηφιακή Επεξεργασία Εικόνας
(επιλογής)}\label{ux3c8ux3b7ux3c6ux3b9ux3b1ux3baux3ae-ux3b5ux3c0ux3b5ux3beux3b5ux3c1ux3b3ux3b1ux3c3ux3afux3b1-ux3b5ux3b9ux3baux3ccux3bdux3b1ux3c2-ux3b5ux3c0ux3b9ux3bbux3bfux3b3ux3aeux3c2}}

\emph{Εισαγωγή στην Ψηφιακή Επεξεργασία Εικόνας. Αναπαράσταση Ψηφιακών
Εικόνων. Στοιχεία ενός Συστήματος Ψηφιακής Επεξεργασίας Εικόνας. Ψηφιακή
Καταγραφή Εικόνας. Τμηματοποίηση και Αυτόματη Ευθυγράμμιση εικόνας.
Εξαγωγή Χαρακτηριστικών και Ανάλυση εικόνας. Ανίχνευση Γραμμών, Δομή,
Σχήμα, Υφή, Ταίριασμα, Τεμάχιση, Κατάταξη. Συμπίεση ψηφιακής εικόνας.
Αλγόριθμοι ανίχνευσης ακμών. Μεθοδολογίες σχεδίασης ψηφιακών φίλτρων.
Bέλτιστα γραμμικά φίλτρα. Αυτοπροσαρμοζόμενα φίλτρα. Στοιχεία Ανθρώπινης
Ορασης. Μοντέλα Εικόνων. Δειγματοληψία και Κβάντιση. Μετασχηματισμός
Εικόνας: Μετασχηματισμός Fourier, DFT, FFT, Walsh, Hadamard, DCT,
Hotelling, Hough. Βελτίωση εικόνας: Τροποποίηση Ιστογράμματος,
Εξομάλυνση, Οξυνση. Αποκατάσταση Εικόνας: Μοντέλο Χειροτέρευσης,
Αλγεβρική Μέθοδος, Αντίστροφο Φιλτράρισμα.}

\hypertarget{ux3baux3b1ux3b9ux3bdux3bfux3c4ux3bfux3bcux3afux3b1-ux3baux3b1ux3b9-ux3b5ux3c0ux3b9ux3c7ux3b5ux3b9ux3c1ux3b7ux3bcux3b1ux3c4ux3b9ux3baux3ccux3c4ux3b7ux3c4ux3b1-ux3b5ux3c0ux3b9ux3bbux3bfux3b3ux3aeux3c2}{%
\subsubsection{Καινοτομία και Επιχειρηματικότητα
(επιλογής)}\label{ux3baux3b1ux3b9ux3bdux3bfux3c4ux3bfux3bcux3afux3b1-ux3baux3b1ux3b9-ux3b5ux3c0ux3b9ux3c7ux3b5ux3b9ux3c1ux3b7ux3bcux3b1ux3c4ux3b9ux3baux3ccux3c4ux3b7ux3c4ux3b1-ux3b5ux3c0ux3b9ux3bbux3bfux3b3ux3aeux3c2}}

\emph{Εισαγωγή στις έννοιες της Επιχειρηματικότητας και της Καινοτομίας.
Eπιχειρηματικό περιβάλλον. Διαδικασία καινοτομίας και δημιουργικότητας.
Μέθοδοι και εργαλεία μέτρησης καινοτομίας. Καινοτομία στην Ελλάδα.
Κλειστή έναντι Ανοικτής Καινοτομίας. Σύλληψη Καινοτόμου Επιχειρηματικής
Ιδέας. Επιλογή Βιώσιμου Επιχειρηματικού Μοντέλου. Επιχειρηματικό Πλάνο:
Ανάπτυξη \& Αξιολόγηση. Ίδρυση της επιχείρησης. Ανεύρεση Πόρων και
Διαμόρφωση Συμφωνιών. Διερεύνηση Στρατηγικών Εξόδου. Η συμβολή της
τεχνολογίας στην ανάπτυξη καινοτομίας. Διεθνής επιχειρηματικότητα και
Μελέτες Περίπτωσης.}

\hypertarget{ux3bcux3bfux3bdux3c4ux3adux3bbux3b1-ux3baux3b2ux3b1ux3bdux3c4ux3b9ux3baux3bfux3cd-ux3baux3b1ux3b9-ux3bcux3bfux3c1ux3b9ux3b1ux3baux3bfux3cd-ux3c5ux3c0ux3bfux3bbux3bfux3b3ux3b9ux3c3ux3bcux3bfux3cd-ux3b5ux3c0ux3b9ux3bbux3bfux3b3ux3aeux3c2}{%
\subsubsection{Μοντέλα Κβαντικού και Μοριακού Υπολογισμού
(επιλογής)}\label{ux3bcux3bfux3bdux3c4ux3adux3bbux3b1-ux3baux3b2ux3b1ux3bdux3c4ux3b9ux3baux3bfux3cd-ux3baux3b1ux3b9-ux3bcux3bfux3c1ux3b9ux3b1ux3baux3bfux3cd-ux3c5ux3c0ux3bfux3bbux3bfux3b3ux3b9ux3c3ux3bcux3bfux3cd-ux3b5ux3c0ux3b9ux3bbux3bfux3b3ux3aeux3c2}}

\emph{Σύντομη εισαγωγή στα κλασικά υπολογιστικά μοντέλα με έμφαση στις
μηχανές Turing. Εισαγωγή σε μη συμβατικά υπολογιστικά μοντέλα. Εισαγωγή
στον μοριακό υπολογισμό. Το πείραμα του Adelman. Λύση δύσκολων
προβλημάτων μέσω του DNA. Εισαγωγή στον Κβαντικό υπολογισμό. Βασικά
στοιχεία κβαντομηχανικής σχετικά με την περιγραφή και τη λειτουργία ενός
φυσικού κβαντικού συστήματος. Ο φορμαλισμός του Dirac. Οι αλγόριθμοι των
Deutsch--Jozsa, του Simon, του Shor και του Grover. Προσομοίωση
κβαντικών συστημάτων υπολογισμού στο Matlab. Ο υπολογιστής D-Wave Two™.}

\hypertarget{ux3b1ux3bdux3b1ux3bbux3c5ux3c4ux3b9ux3baux3ae-ux3b4ux3b5ux3b4ux3bfux3bcux3adux3bdux3c9ux3bd-ux3c5ux3b3ux3b5ux3afux3b1ux3c2-ux3b5ux3c0ux3b9ux3bbux3bfux3b3ux3aeux3c2}{%
\subsubsection{Αναλυτική Δεδομένων Υγείας
(επιλογής)}\label{ux3b1ux3bdux3b1ux3bbux3c5ux3c4ux3b9ux3baux3ae-ux3b4ux3b5ux3b4ux3bfux3bcux3adux3bdux3c9ux3bd-ux3c5ux3b3ux3b5ux3afux3b1ux3c2-ux3b5ux3c0ux3b9ux3bbux3bfux3b3ux3aeux3c2}}

\emph{To μάθημα Αναλυτική Δεδομένων Υγείας έχει ως στόχο να μάθει τους
φοιτητές θεωρία και πράξη πάνω στην αναλυτική δεδομένων και κυρίως όσον
αφορά στα δεδομένα υγείας και ιατρικής. Από τις βασικές έννοιες ανάλυσης
δεδομένων υγείας μέχρι το σχεδιασμό τεχνικών και εφαρμογών από πλευράς
τεχνολογία λογισμικού, την υλοποίηση τους και την αξιολόγηση τους. Τέλος
θα παρουσιαστούν πραγματικά παραδείγματα.}

\hypertarget{ux3bfux3c0ux3c4ux3b9ux3baux3bfux3c0ux3bfux3afux3b7ux3c3ux3b7-ux3c4ux3b7ux3c2-ux3c0ux3bbux3b7ux3c1ux3bfux3c6ux3bfux3c1ux3afux3b1ux3c2-ux3b5ux3c0ux3b9ux3bbux3bfux3b3ux3aeux3c2}{%
\subsubsection{Οπτικοποίηση της Πληροφορίας
(επιλογής)}\label{ux3bfux3c0ux3c4ux3b9ux3baux3bfux3c0ux3bfux3afux3b7ux3c3ux3b7-ux3c4ux3b7ux3c2-ux3c0ux3bbux3b7ux3c1ux3bfux3c6ux3bfux3c1ux3afux3b1ux3c2-ux3b5ux3c0ux3b9ux3bbux3bfux3b3ux3aeux3c2}}

\emph{Οπτικοποίηση της πληροφορίας και διάδραση, ορισμός και ιστορική
εξέλιξη, σχεδίαση και αισθητική της οπτικοποίησης. Τεχνολογίες
οπτικοποίησης: Διάδραση, Γραφικά, Πολυμέσα, Μεγάλα δεδομένα. Σύγχρονες
εφαρμογές οπτικοποίησης: Γεωγραφικοί χάρτες, ψηφιακή δημοσιογραφία,
δεδομένα υγείας-ευζωϊας, προσωπικά δεδομένα, λήψη αποφάσεων, πολιτισμός-
τέχνη, ψηφιακές συλλογές. Ανάπτυξη μιας επίκαιρης εφαρμογής
οπτικοποίησης δεδομένων με σύγχρονα εργαλεία του ιστού και
διασυνδεδεμένα δεδομένα.}

\hypertarget{ux3b5ux3beux3acux3bcux3b7ux3bdux3bf-ux3c3ux3c4}{%
\subsection{Εξάμηνο
ΣΤ}\label{ux3b5ux3beux3acux3bcux3b7ux3bdux3bf-ux3c3ux3c4}}

\hypertarget{ux3c4ux3b5ux3c7ux3bdux3bfux3bbux3bfux3b3ux3afux3b1-ux3bbux3bfux3b3ux3b9ux3c3ux3bcux3b9ux3baux3bfux3cd-ux3baux3bfux3c1ux3bcux3bfux3cd}{%
\subsubsection{Τεχνολογία Λογισμικού
(κορμού)}\label{ux3c4ux3b5ux3c7ux3bdux3bfux3bbux3bfux3b3ux3afux3b1-ux3bbux3bfux3b3ux3b9ux3c3ux3bcux3b9ux3baux3bfux3cd-ux3baux3bfux3c1ux3bcux3bfux3cd}}

\emph{Κύκλος ζωής λογισμικού. Μεθοδολογίες ανάπτυξης λογισμικού.
Σχεδιασμός και αρχιτεκτονική συστήματος. Κατασκευή διεπαφής χρήστη.
Διαδικασία παράδοσης και συντήρησης συστημάτων λογισμικού. Συνεργατικά
συστήματα. Ψυχαγωγικό και Εκπαιδευτικό Λογισμικό.}

\hypertarget{ux3c4ux3b5ux3c7ux3bdux3b7ux3c4ux3ae-ux3bdux3bfux3b7ux3bcux3bfux3c3ux3cdux3bdux3b7-ux3baux3bfux3c1ux3bcux3bfux3cd}{%
\subsubsection{Τεχνητή Νοημοσύνη
(κορμού)}\label{ux3c4ux3b5ux3c7ux3bdux3b7ux3c4ux3ae-ux3bdux3bfux3b7ux3bcux3bfux3c3ux3cdux3bdux3b7-ux3baux3bfux3c1ux3bcux3bfux3cd}}

\emph{Στόχοι της Τεχνητής Νοημοσύνης. Ιστορική Αναδρομή. Προβλήματα και
επίλυση. Τεχνικές Αναζήτησης. Τυφλή και πληροφορημένη αναζήτηση.
Αναζήτηση λύσης σε παιχνίδια δύο αντιπάλων. Προτασιακή Λογική.
Κατηγορηματική Λογική. Κανόνες Συμπερασμού. Συλλογιστική. Αναπαράσταση
Γνώσης. Σημασιολογικά Δίκτυα. Εννοιολογικοί Γράφοι. Μηχανική Μάθηση.
Μάθηση με βάση τα παραδείγματα. Οι αλγόριθμοι του πλησιέστερου γείτονα.
Δέντρα Αποφάσεων. Στοχαστική Μάθηση. Η πλατφόρμα μηχανικής μάθησης Weka.
Έμπειρα Συστήματα. Η Γλώσσα παραγωγής CLIPS. Εφαρμογές Τεχνητής
Νοημοσύνης.}

\hypertarget{ux3c4ux3b5ux3c7ux3bdux3bfux3bbux3bfux3b3ux3afux3b5ux3c2-ux3b4ux3b9ux3b1ux3b4ux3b9ux3baux3c4ux3cdux3bfux3c5-ux3baux3b1ux3c4ux3b5ux3cdux3b8ux3c5ux3bdux3c3ux3b7ux3c2-ux3c0ux3c3}{%
\subsubsection{Τεχνολογίες Διαδικτύου (κατεύθυνσης
ΠΣ)}\label{ux3c4ux3b5ux3c7ux3bdux3bfux3bbux3bfux3b3ux3afux3b5ux3c2-ux3b4ux3b9ux3b1ux3b4ux3b9ux3baux3c4ux3cdux3bfux3c5-ux3baux3b1ux3c4ux3b5ux3cdux3b8ux3c5ux3bdux3c3ux3b7ux3c2-ux3c0ux3c3}}

\emph{Προγραμματισμός στον πελάτη (Client-side programming): HTML, HTML5
και JavaScript. Προγραμματισμός στον εξυπηρετητή (Server-side
programming): Web Servers, δομή και λειτουργία. Η γλώσσα PHP. Βάσεις
δεδομένων στο Διαδίκτυο: MySQL, σύνδεση με Apache Web Server,
PHP/Python. Web services. Πρωτόκολλο επικοινωνίας SOAP. Μεταδεδομένα
στον παγκόσμιο ιστό: XML-JSON. Υπολογιστικά Νέφη (Cloud Computing) και
υπηρεσίες τους (Software-as-a-Service -- SaaS). Google AppEngine.
Επιθέσεις και Ασφάλεια στον Παγκόσμιο Ιστό. Web 2.0, 3.0.}

\hypertarget{ux3c3ux3c5ux3c3ux3c4ux3aeux3bcux3b1ux3c4ux3b1-ux3c5ux3c0ux3bfux3c3ux3c4ux3aeux3c1ux3b9ux3beux3b7ux3c2-ux3b1ux3c0ux3bfux3c6ux3acux3c3ux3b5ux3c9ux3bd-ux3baux3b1ux3c4ux3b5ux3cdux3b8ux3c5ux3bdux3c3ux3b7ux3c2-ux3c0ux3c3}{%
\subsubsection{Συστήματα Υποστήριξης Αποφάσεων (κατεύθυνσης
ΠΣ)}\label{ux3c3ux3c5ux3c3ux3c4ux3aeux3bcux3b1ux3c4ux3b1-ux3c5ux3c0ux3bfux3c3ux3c4ux3aeux3c1ux3b9ux3beux3b7ux3c2-ux3b1ux3c0ux3bfux3c6ux3acux3c3ux3b5ux3c9ux3bd-ux3baux3b1ux3c4ux3b5ux3cdux3b8ux3c5ux3bdux3c3ux3b7ux3c2-ux3c0ux3c3}}

\emph{Εισαγωγή στη θεωρία Αποφάσεων, Η φιλοσοφία των Συστημάτων
Υποστήριξης Αποφάσεων και ο ρόλος τους στις διαδικασίες λήψης αποφάσεων
στην επιχείρηση, Αρχιτεκτονική Συστημάτων Υποστήριξης Αποφάσεων,
Συστημάτων Υποστήριξης Αποφάσεων βασισμένα στη διαχείριση Βάσεων
Δεδομένων και Συστημάτων Υποστήριξης Αποφάσεων βασισμένα στη διαχείριση
Βάσεων Μοντέλων, Τεχνικές και Μοντέλα λήψης αποφάσεων: Δέντρα αποφάσεων,
μαθηματικός προγραμματισμός, ανάλυση ευαισθησίας, what-if ανάλυση,
ανάλυση βάσει στόχων. Πολυκριτηριακά Συστημάτων Υποστήριξης Αποφάσεων,
Συστήματα υποστήριξης ομαδικής λήψης αποφάσεων (GDSS). Αξιοποίηση
Αποθηκών Δεδομένων (Data Warehouses) και τεχνικών Εξόρυξης Δεδομένων
(Data Mining) για την υποστήριξη της λήψης αποφάσεων, Συστήματα
πληροφόρησης και υποστήριξης ανωτέρων στελεχών (EIS-ESS), Γεωγραφικά
Πληροφοριακά Συστήματα (GIS). Βασισμένα στη γνώση (KMS) και Έμπειρα
Συστήματα (ES), Εφαρμογές και παραδείγματα ΣΥΑ.}

\hypertarget{ux3b1ux3bdux3b1ux3b3ux3bdux3ceux3c1ux3b9ux3c3ux3b7-ux3c0ux3c1ux3bfux3c4ux3cdux3c0ux3c9ux3bd-ux3baux3b1ux3c4ux3b5ux3cdux3b8ux3c5ux3bdux3c3ux3b7ux3c2-ux3c0ux3b1ux3baux3b5}{%
\subsubsection{Αναγνώριση Προτύπων (κατεύθυνσης
ΠΑΚΕ)}\label{ux3b1ux3bdux3b1ux3b3ux3bdux3ceux3c1ux3b9ux3c3ux3b7-ux3c0ux3c1ux3bfux3c4ux3cdux3c0ux3c9ux3bd-ux3baux3b1ux3c4ux3b5ux3cdux3b8ux3c5ux3bdux3c3ux3b7ux3c2-ux3c0ux3b1ux3baux3b5}}

\emph{Μέθοδοι και συστήματα αναγνώρισης προτύπων. Όρια στην ακρίβεια
μέτρησης της αξιοπιστίας αναγνώρισης. Κατευθυνόμενη εκπαίδευση και
αυτοεκπαίδευση. Συναρτήσεις απόστασης. Ταξινόμηση με κριτήριο την
μικρότερη απόσταση και τα κοντινότερα πρότυπα. Γραμμικές και μη
γραμμικές συναρτήσεις απόφασης. Ο αλγόριθμος Perceptron. Ταξινομητές
Bayes, ταξινομητές πλησιέστερου γείτονα. Παραμετρική και μη παραμετρική
εκτίμηση της πυκνότητας πιθανότητας προτύπων: Μεγιστοποίηση εντροπίας,
εκτιμητής Parzen, ορθοκανονικές συναρτήσεις, μέθοδοι των RobbinsMonro
και KieferWolfowitz, LMS. Μέθοδοι ελαχίστων τετραγώνων. Πολυστρωματικά
τεχνητά νευρωνικά δίκτυα. Aναδρομικά τεχνητά νευρωνικά δίκτυα.
Εκπαίδευση διόρθωσης λάθους, Hebbian και ανταγωνιστική εκπαίδευση.
Πολυεπίπεδο perceptron. Οπισθοδρομική διάδοση του σφάλματος. Δίκτυα
ακτινικών συναρτήσεων. Μηχανή Hopfield. Μάθηση με και χωρίς επιτήρηση.
Ιεραρχική ομαδοποίηση δεδομένων. Ασαφής λογική. Γενετικοί αλγόριθμοι και
αρχές εξελικτικού υπολογισμού.}

\hypertarget{ux3b1ux3bdux3acux3baux3c4ux3b7ux3c3ux3b7-ux3c0ux3bbux3b7ux3c1ux3bfux3c6ux3bfux3c1ux3afux3b1ux3c2-ux3baux3b1ux3c4ux3b5ux3cdux3b8ux3c5ux3bdux3c3ux3b7ux3c2-ux3c0ux3b1ux3baux3b5}{%
\subsubsection{Ανάκτηση Πληροφορίας (κατεύθυνσης
ΠΑΚΕ)}\label{ux3b1ux3bdux3acux3baux3c4ux3b7ux3c3ux3b7-ux3c0ux3bbux3b7ux3c1ux3bfux3c6ux3bfux3c1ux3afux3b1ux3c2-ux3baux3b1ux3c4ux3b5ux3cdux3b8ux3c5ux3bdux3c3ux3b7ux3c2-ux3c0ux3b1ux3baux3b5}}

\emph{Το μάθημα εισάγει τους φοιτητές στις έννοιες της λογικής
αναπαράστασης κειμένων και τους εξοικειώνει με τα βασικά μοντέλα και
διαδικασίες ανάκτησης πληροφοριών ούτως ώστε να αποκτήσουν μία συνολική
αντίληψη επ 'αυτών. Με την επιτυχή ολοκλήρωση του μαθήματος οι φοιτητές
θα έχουν κατανοήσει τα βασικά χαρακτηριστικά των τεχνικών ανάκτησης
πληροφορίας, καθώς και θα έχουν λάβει γνώση των βασικών εργαλείων,
αλγορίθμων και μεθοδολογιών με τα οποία αυτή υλοποιείται στα σημερινά
υπολογιστικά περιβάλλοντα.}

\hypertarget{ux3baux3b1ux3c4ux3b1ux3bdux3b5ux3bcux3b7ux3bcux3adux3bdux3b1-ux3b4ux3b9ux3baux3c4ux3c5ux3bfux3baux3b5ux3bdux3c4ux3c1ux3b9ux3baux3ac-ux3c3ux3c5ux3c3ux3c4ux3aeux3bcux3b1ux3c4ux3b1-ux3b5ux3c0ux3b9ux3bbux3bfux3b3ux3aeux3c2}{%
\subsubsection{Κατανεμημένα Δικτυοκεντρικά Συστήματα
(επιλογής)}\label{ux3baux3b1ux3c4ux3b1ux3bdux3b5ux3bcux3b7ux3bcux3adux3bdux3b1-ux3b4ux3b9ux3baux3c4ux3c5ux3bfux3baux3b5ux3bdux3c4ux3c1ux3b9ux3baux3ac-ux3c3ux3c5ux3c3ux3c4ux3aeux3bcux3b1ux3c4ux3b1-ux3b5ux3c0ux3b9ux3bbux3bfux3b3ux3aeux3c2}}

\emph{Την εποχή του Διαδικτύου, των διάφορων ενσύρματων και ασύρματων
δικτύων και των φορητών υπολογιστών πολλαπλών πυρήνων, η σημασία των
κατανεμημένων συστημάτων είναι καθοριστική. Τα κατανεμημένα συστήματα
και οι κατανεμημένοι αλγόριθμοι είναι πολύ διαφορετικοί και πολύ πιο
περίπλοκοι, επειδή οι εκτελέσεις στους κόμβους σε ένα κατανεμημένο
σύστημα αλληλοεπικαλύπτονται. Όταν δύο κόμβοι μπορούν να εκτελούν
ταυτόχρονα συμβάντα, δεν μπορεί να προβλεφθεί ποιο από τα συμβάντα θα
συμβεί πρώτα στον χρόνο. Αυτό δημιουργεί, για παράδειγμα, τις λεγόμενες
συνθήκες ανταγωνισμού. Εάν δύο μηνύματα ταξιδεύουν στον ίδιο κόμβο στο
δίκτυο, ενδέχεται να προκύψει διαφορετική συμπεριφορά ανάλογα με το ποια
από τα μηνύματα φτάνουν πρώτα στον προορισμό τους. Τα κατανεμημένα
συστήματα είναι επομένως εγγενώς μη ντετερμινιστικά: η εκτέλεση ενός
συστήματος δύο φορές από την ίδια αρχική διαμόρφωση μπορεί να αποφέρει
διαφορετικά αποτελέσματα. Μια άλλη σημαντική διάκριση είναι ότι στα
κατανεμημένα συστήματα, οι κόμβοι συνήθως γνωρίζουν μόνο τη δική τους
τοπική κατάσταση και όχι του συστήματος, με αποτέλεσμα η ανίχνευση
τερματισμού να γίνεται ζήτημα, καθώς πρέπει να προσδιοριστεί ότι όλοι οι
κόμβοι του συστήματος έχουν τερματιστεί.}

\hypertarget{ux3b1ux3c3ux3c6ux3acux3bbux3b5ux3b9ux3b1-ux3b4ux3b9ux3baux3c4ux3cdux3c9ux3bd-ux3c5ux3c0ux3bfux3bbux3bfux3b3ux3b9ux3c3ux3c4ux3ceux3bd-ux3baux3b1ux3b9-ux3b5ux3c0ux3b9ux3baux3bfux3b9ux3bdux3c9ux3bdux3b9ux3ceux3bd-ux3b5ux3c0ux3b9ux3bbux3bfux3b3ux3aeux3c2}{%
\subsubsection{Ασφάλεια Δικτύων Υπολογιστών και Επικοινωνιών
(επιλογής)}\label{ux3b1ux3c3ux3c6ux3acux3bbux3b5ux3b9ux3b1-ux3b4ux3b9ux3baux3c4ux3cdux3c9ux3bd-ux3c5ux3c0ux3bfux3bbux3bfux3b3ux3b9ux3c3ux3c4ux3ceux3bd-ux3baux3b1ux3b9-ux3b5ux3c0ux3b9ux3baux3bfux3b9ux3bdux3c9ux3bdux3b9ux3ceux3bd-ux3b5ux3c0ux3b9ux3bbux3bfux3b3ux3aeux3c2}}

\emph{Βασικές αρχές κρυπτογραφίας. Εξασφάλιση εμπιστευτικότητας,
ακεραιότητας δεδομένων και αυθεντικοποίηση. Επιθέσεις αντανάκλασης και
ενδιαμέσου. Συστήματα συμφωνίας και διανομής κλειδιού. Tο πρωτόκολλο
Kerberos. Ασφάλεια Ασύρματων δικτύων και πρωτόκολλα ασφάλειας WPA/WPA2.
Ασφάλεια σε κινητά δίκτυα GSM/UMTS/LTE. Μοντέλο απειλών στο Επίπεδο
TCP/IP. Ανάλυση του πρωτοκόλλου TLS/SSL. Ανάλυση του μηχανισμού Pretty
Good Privacy. Ανάλυση, εφαρμογή και αξιολόγηση των αναχωμάτων ασφάλειας
(Firewalls) και εικονικών δικτύων (Virtual Private Networks). Επιθέσεις
παρεισφρήσεων (intrusions) και μηχανισμοί ανίχνευση εισβολών (intrusion
detection systems) σε δικτυακά συστήματα, Επιθέσεις άρνησης παροχής
υπηρεσιών στο Διαδίκτυο και τρόποι αντιμετώπισης τους. Επιθέσεις στο
πρωτόκολλο Domain Name System (DNS) και Address Resolution Protocol
(ARP). Εισαγωγή στο κακόβουλο λογισμικό διαδικτύου και botnets.}

\hypertarget{ux3c3ux3c4ux3bfux3c7ux3b1ux3c3ux3c4ux3b9ux3baux3ae-ux3b1ux3bdux3acux3bbux3c5ux3c3ux3b7-ux3b4ux3b5ux3b4ux3bfux3bcux3adux3bdux3c9ux3bd-ux3b5ux3c0ux3b9ux3bbux3bfux3b3ux3aeux3c2}{%
\subsubsection{Στοχαστική ανάλυση δεδομένων
(επιλογής)}\label{ux3c3ux3c4ux3bfux3c7ux3b1ux3c3ux3c4ux3b9ux3baux3ae-ux3b1ux3bdux3acux3bbux3c5ux3c3ux3b7-ux3b4ux3b5ux3b4ux3bfux3bcux3adux3bdux3c9ux3bd-ux3b5ux3c0ux3b9ux3bbux3bfux3b3ux3aeux3c2}}

\emph{Ειδικά Θέματα Πιθανοτήτων, Εισαγωγή στις χρονοσειρές,
Χαρακτηριστικά χρονοσειρών, Στασιμότητα, Συσχέτιση, Στοχαστικός Θόρυβος,
Είδη θορύβου, Τεχνικές απαλοιφής Θορύβου, Διαδικασίες μέσου όρου,
Βασικές στοχαστικές διαδικασίες, Τυχαίος Περιπατητής, Διαδικασία
Ornstein--Uhlenbeck, Εισαγωγή στις προσομοιώσεις στοχαστικών διαφορικών
εξισώσεων, Μέθοδος Euler--Maruyama, Υπολογιστική Μοντελοποίηση, Fractal,
Μορφοκλασματικές διαδικασίες, Μνήμη χρονοσειρών Στόχος του μαθήματος
αποτελεί : Η εξοικείωση με μεθόδους ανάλυσης χρονοσειρών και η ανάπτυξη
εργαλείων και λογισμικού για την ανάλυση και οπτικοποίηση τους. Η
θεωρητική κατάρτιση σε μεθόδους στοχαστικών διαδικασιών, προσομοιώσεων
και μοντελοποίησης με σκοπό την ανάλυση δεδομένων. Η δημιουργία
συνθετικών δεδομένων για πειραματισμό μέσω προσομοιώσεων και ανάπτυξη
δεξιοτήτων στην συγγραφή επιστημονικού κώδικα. Εφαρμογές σε πραγματικά
δεδομένα πεδίου.}

\hypertarget{ux3b2ux3b9ux3bfux3c0ux3bbux3b7ux3c1ux3bfux3c6ux3bfux3c1ux3b9ux3baux3ae-ux3b5ux3c0ux3b9ux3bbux3bfux3b3ux3aeux3c2}{%
\subsubsection{Βιοπληροφορική
(επιλογής)}\label{ux3b2ux3b9ux3bfux3c0ux3bbux3b7ux3c1ux3bfux3c6ux3bfux3c1ux3b9ux3baux3ae-ux3b5ux3c0ux3b9ux3bbux3bfux3b3ux3aeux3c2}}

\emph{- Μοντέλα αναπαράστασης βιολογικών μορίων, μαθηματική
μοντελοποίηση στα γονιδιακά ρυθμιστικά δίκτυα, οντολογίες, στοίχιση
ακολουθιών, αλγόριθμοι στη μοριακή βιολογία, αλγόριθμοι στη δομική
πληροφορική, δομή και κατασκευή βιολογικών μοντέλων, μετάφραση ενός
βιολογικού ερωτήματος σε ένα μαθηματικό μοντέλο, ποιοτικά και ποσοτικά
μοντέλα, προσδιοριστικά μοντέλα, ανάλυση αποτελεσμάτων, επικύρωση και
επαλήθευση, τεχνική ταυτοποίηση μοντέλων δεδομένων,αναδρομικοί
αλγόριθμοι, επιλογή κατάλληλου μοντέλου, ομαδοποίηση, αλγόριθμοι
ομαδοποίησης, μηχανική μάθηση, ταξινόμηση, νευρωνικά δίκτυα.}

\hypertarget{ux3c0ux3c1ux3b1ux3baux3c4ux3b9ux3baux3ae-ux3acux3c3ux3baux3b7ux3c3ux3b7-ux3b5ux3c0ux3b9ux3bbux3bfux3b3ux3aeux3c2}{%
\subsubsection{Πρακτική άσκηση
(επιλογής)}\label{ux3c0ux3c1ux3b1ux3baux3c4ux3b9ux3baux3ae-ux3acux3c3ux3baux3b7ux3c3ux3b7-ux3b5ux3c0ux3b9ux3bbux3bfux3b3ux3aeux3c2}}

\emph{Η Πρακτική Άσκηση δίνει στις φοιτήτριες και τους φοιτητές τη
δυνατότητα να αντιμετωπίσουν πραγματικά προβλήματα που σχετίζονται με
την επιστήμη τους στην αγορά εργασίας και να τα επιλύσουν
χρησιμοποιώντας στην πράξη διδαχθείσες μεθόδους και τεχνολογίες, καθώς
επίσης και να εξοικειωθούν με εξοπλισμό που μετά το πέρας των σπουδών
τους είναι δυνατό να κληθούν να χρησιμοποιήσουν.}

\hypertarget{ux3b5ux3beux3acux3bcux3b7ux3bdux3bf-ux3b6}{%
\subsection{Εξάμηνο Ζ}\label{ux3b5ux3beux3acux3bcux3b7ux3bdux3bf-ux3b6}}

\hypertarget{ux3b7ux3bbux3b5ux3baux3c4ux3c1ux3bfux3bdux3b9ux3baux3cc-ux3b5ux3c0ux3b9ux3c7ux3b5ux3b9ux3c1ux3b5ux3afux3bd-ux3baux3bfux3c1ux3bcux3bfux3cd}{%
\subsubsection{Ηλεκτρονικό Επιχειρείν
(κορμού)}\label{ux3b7ux3bbux3b5ux3baux3c4ux3c1ux3bfux3bdux3b9ux3baux3cc-ux3b5ux3c0ux3b9ux3c7ux3b5ux3b9ux3c1ux3b5ux3afux3bd-ux3baux3bfux3c1ux3bcux3bfux3cd}}

\emph{Εισαγωγή στην Ψηφιακή Οικονομία και το Ηλεκτρονικό Επιχειρείν.
Βασικοί Ορισμοί. Ηλεκτρονικό Εμπόριο vs.~Ηλεκτρονικό Επιχειρείν. Μοντέλα
και Εφαρμογές Διεπιχειρησιακού (B2B) Ηλεκτρονικού Επιχειρείν. Μοντέλα
και Εφαρμογές Πελατοκεντρικού (B2C) Ηλεκτρονικού Επιχειρείν. Άλλες
Εφαρμογές Αξίας στην Ψηφιακή Οικονομία (Εταιρικές Πύλες, Ηλεκτρονική
Διακυβέρνηση, Διαχείριση Σχέσεων με Πελάτες - CRM). Ηλεκτρονικά
Συστήματα Πληρωμών (e-Payment). Τεχνολογίες και Δικτυακές Υποδομές
(Intranets/Extranets, Δίκτυα VPN). Ηλεκτρονικό Μάρκετινγκ και
Επικοινωνία. Ασφάλεια και Προστασία στο Ηλεκτρονικό Επιχειρείν.
Νομοθετικό Πλαίσιο και Ηθική στο Ηλεκτρονικό Επιχειρείν. Στρατηγική
Διαχείριση Ηλεκτρονικού Επιχειρείν. Νέες Μορφές Ηλεκτρονικού Επιχειρείν
(Κινητό και Ασύρματο Επιχειρείν).}

\hypertarget{ux3c0ux3c1ux3bfux3c3ux3bfux3bcux3bfux3afux3c9ux3c3ux3b7-ux3baux3b1ux3b9-ux3bcux3bfux3bdux3c4ux3b5ux3bbux3bfux3c0ux3bfux3afux3b7ux3c3ux3b7-ux3baux3b1ux3c4ux3b5ux3cdux3b8ux3c5ux3bdux3c3ux3b7ux3c2-ux3c0ux3c3}{%
\subsubsection{Προσομοίωση και Μοντελοποίηση (κατεύθυνσης
ΠΣ)}\label{ux3c0ux3c1ux3bfux3c3ux3bfux3bcux3bfux3afux3c9ux3c3ux3b7-ux3baux3b1ux3b9-ux3bcux3bfux3bdux3c4ux3b5ux3bbux3bfux3c0ux3bfux3afux3b7ux3c3ux3b7-ux3baux3b1ux3c4ux3b5ux3cdux3b8ux3c5ux3bdux3c3ux3b7ux3c2-ux3c0ux3c3}}

\emph{Προσομοίωση και εξομοίωση. Δομή και κατασκευή μοντέλων
προσομοίωσης. Παραγωγή τυχαίων αριθμών και τυχαίων μεταβλητών.
Μηχανισμοί ροής χρόνου. Στοχαστικά μοντέλα αλληλεπιδραστικής
προσομοίωσης. Προσομοίωση γεγονότων, προσομοίωση δραστηριοτήτων. Γλώσσες
προσομοίωσης. Ανάπτυξη προγραμμάτων προσομοίωσης, εξειδικευμένες γλώσσες
προσομοίωσης. Ανάλυση αποτελεσμάτων, επικύρωση και επαλήθευση των
αποτελεσμάτων. Τεχνικές προσδιορισμού μαθηματικών μοντέλων από
δεδομένα-μετρήσεις κρίσιμων μεγεθών του συστήματος/διαδικασίας. Μοντέλα
δυναμικών συστημάτων, μοντελοποίηση ως μαύρο κουτί, αναδρομικοί
αλγόριθμοι προσδιορισμού παραμέτρων του μοντέλου, αξιολόγηση μοντέλου,
προεπεξεργασία δεδομένων, πρακτικά θέματα αναγνώρισης συστημάτων.}

\hypertarget{ux3c0ux3bbux3b7ux3c1ux3bfux3c6ux3bfux3c1ux3b9ux3b1ux3baux3ac-ux3c3ux3c5ux3c3ux3c4ux3aeux3bcux3b1ux3c4ux3b1-ux3baux3b1ux3b9-ux3b5ux3c6ux3bfux3b4ux3b9ux3b1ux3c3ux3c4ux3b9ux3baux3ae-ux3b1ux3bbux3c5ux3c3ux3afux3b4ux3b1-ux3baux3b1ux3c4ux3b5ux3cdux3b8ux3c5ux3bdux3c3ux3b7ux3c2-ux3c0ux3c3}{%
\subsubsection{Πληροφοριακά Συστήματα και Εφοδιαστική Αλυσίδα
(κατεύθυνσης
ΠΣ)}\label{ux3c0ux3bbux3b7ux3c1ux3bfux3c6ux3bfux3c1ux3b9ux3b1ux3baux3ac-ux3c3ux3c5ux3c3ux3c4ux3aeux3bcux3b1ux3c4ux3b1-ux3baux3b1ux3b9-ux3b5ux3c6ux3bfux3b4ux3b9ux3b1ux3c3ux3c4ux3b9ux3baux3ae-ux3b1ux3bbux3c5ux3c3ux3afux3b4ux3b1-ux3baux3b1ux3c4ux3b5ux3cdux3b8ux3c5ux3bdux3c3ux3b7ux3c2-ux3c0ux3c3}}

\emph{Εισαγωγή στη διαχείριση της εφοδιαστικής αλυσίδας.Τυπικά
προβλήματα στη διαχείριση της εφοδιαστικής αλυσίδας. Τεχνικές
διαχείρισης αποθεμάτων και συνεργασίας στις σύγχρονες εφοδιαστικές
αλυσίδες. Λειτουργικότητα και αρχιτεκτονική ERP
συστημάτων.Λειτουργικότητα και αρχιτεκτονική CRM συστημάτων. Μεθοδολογία
διαχείρισης και προμήθειας ERP συστημάτων. Αρχές μοντελοποίησης και
κοστολόγησης επιχειρηματικών διαδικασιών. Μετασχηματισμός
επιχειρηματικών διαδικασιών και εφοδιαστικής αλυσίδας με τη χρήση
πληροφορικής. Σύγχρονες τεχνολογικές τάσεις διαχείρισης εφοδιαστικής
αλυσίδας (RFID, cloud computing, Internet of things).}

\hypertarget{ux3b3ux3bbux3c9ux3c3ux3c3ux3b9ux3baux3ae-ux3c4ux3b5ux3c7ux3bdux3bfux3bbux3bfux3b3ux3afux3b1-ux3baux3b1ux3c4ux3b5ux3cdux3b8ux3c5ux3bdux3c3ux3b7ux3c2-ux3c0ux3b1ux3baux3b5}{%
\subsubsection{Γλωσσική Τεχνολογία (κατεύθυνσης
ΠΑΚΕ)}\label{ux3b3ux3bbux3c9ux3c3ux3c3ux3b9ux3baux3ae-ux3c4ux3b5ux3c7ux3bdux3bfux3bbux3bfux3b3ux3afux3b1-ux3baux3b1ux3c4ux3b5ux3cdux3b8ux3c5ux3bdux3c3ux3b7ux3c2-ux3c0ux3b1ux3baux3b5}}

\emph{Υπολογιστική Γλωσσολογία και Επεξεργασία Φυσικής Γλώσσας. Τα
χαρακτηριστικά της φυσικής γλώσσας. Μορφολογική επεξεργασία. Κανονικές
Εκφράσεις. Αυτόματα και Μετατροπείς Πεπερασμένων Καταστάσεων. Σύνταξη.
Ανάπτυξη Γραμματικών. Τύποι Γραμματικών και φορμαλισμοί. Iεραρχία
Chomsky. Συντακτική Ανάλυση. Σημασιολογική Επεξεργασία. Ερμηνεία. Λογική
Φόρμα. Επιλεκτικοί περιορισμοί. Σημασιολογικά δίκτυα. Οντολογίες.
Πραγματολογία. Ανάλυση Λόγου. Επίλυση αναφορών. Επισημείωση μερών του
λόγου. Στοχαστική σύνταξη. Επαγωγή γραμματικής. Άρση Αμφισημίας Λέξεων.
Σύνθεση Φυσικής Γλώσσας. Αυτόματη Μετάφραση. Εξαγωγή Πληροφορίας. Το
πακέτο εργαλείων επεξεργασίας φυσικής γλώσσας NLTK. Εφαρμογές μηχανικής
μάθησης στην επεξεργασία φυσικής γλώσσας.}

\hypertarget{ux3c4ux3b5ux3c7ux3bdux3bfux3bbux3bfux3b3ux3afux3b1-ux3c8ux3c5ux3c7ux3b1ux3b3ux3c9ux3b3ux3b9ux3baux3bfux3cd-ux3bbux3bfux3b3ux3b9ux3c3ux3bcux3b9ux3baux3bfux3cd-ux3b5ux3b9ux3baux3bfux3bdux3b9ux3baux3bfux3af-ux3baux3ccux3c3ux3bcux3bfux3b9-ux3baux3b1ux3c4ux3b5ux3cdux3b8ux3c5ux3bdux3c3ux3b7ux3c2-ux3c0ux3b1ux3baux3b5}{%
\subsubsection{Τεχνολογία Ψυχαγωγικού Λογισμικού \& Εικονικοί Κόσμοι
(κατεύθυνσης
ΠΑΚΕ)}\label{ux3c4ux3b5ux3c7ux3bdux3bfux3bbux3bfux3b3ux3afux3b1-ux3c8ux3c5ux3c7ux3b1ux3b3ux3c9ux3b3ux3b9ux3baux3bfux3cd-ux3bbux3bfux3b3ux3b9ux3c3ux3bcux3b9ux3baux3bfux3cd-ux3b5ux3b9ux3baux3bfux3bdux3b9ux3baux3bfux3af-ux3baux3ccux3c3ux3bcux3bfux3b9-ux3baux3b1ux3c4ux3b5ux3cdux3b8ux3c5ux3bdux3c3ux3b7ux3c2-ux3c0ux3b1ux3baux3b5}}

\emph{Ιστορία και εξέλιξη του ψυχαγωγικού λογισμικού (βιντεοπαιχνιδιών).
Η βιομηχανία ψυχαγωγικού λογισμικού σήμερα, τα είδη των παιχνιδιών, το
προφίλ του χρήστη, διαθέσιμες πλατφόρμες παιχνιδιών, επιχειρηματικά
μοντέλα και μάρκετινγκ. Internet και ψυχαγωγικό λογισμικό, διαδικτυακά
παιχνίδια πολλών χρηστών (MMOGs), η οικονομία και παραοικονομία τους.
Αρχές θεωρίας σχεδιασμού ψυχαγωγικού λογισμικού, οι φάσεις ανάπτυξης, η
ομάδα ανάπτυξης και ειδικότητες. Προγραμματισμός ψυχαγωγικού λογισμικού,
διαθέσιμα εργαλεία, διαφορές για κάθε πλατφόρμα. Σύγχρονες τεχνικές
γραφικών και τεχνητή νοημοσύνη στα παιχνίδια. Μηχανές ψυχαγωγικού
λογισμικού. Ανάπτυξη και διαχείριση ψηφιακού περιεχομένου. Εναλλακτικές
εφαρμογές ψυχαγωγικού λογισμικού. Κοινωνικός αντίκτυπος, θέματα εθισμού
και βίας στα βιντεοπαιχνίδια.Συστήματα εικονικών περιβαλλόντων (virtual
environment systems), εικονικοί κόσμοι (virtual worlds), Περιβάλλοντα
εµβύθισης (immersive environments), Περιβάλλοντα οθόνης (desktop
environments), Περιβάλλοντα προβολής (projected environments),
Ενισχυμένα περιβάλλοντα (augmented environments), κύβος Αυτονομίας --
Αλληλεπίδρασης -- Παρουσίας (Autonomy -- Interaction -- Presence, AIP
cube), χώρος σκηνικού, εικονικοί ηθοποιοί.}

\hypertarget{ux3baux3bfux3b9ux3bdux3c9ux3bdux3b9ux3baux3ac-ux3baux3b1ux3b9-ux3bdux3bfux3bcux3b9ux3baux3ac-ux3b8ux3adux3bcux3b1ux3c4ux3b1-ux3c4ux3c9ux3bd-ux3c4ux3c0ux3b5-ux3b5ux3c0ux3b9ux3bbux3bfux3b3ux3aeux3c2}{%
\subsubsection{Κοινωνικά και Νομικά Θέματα των ΤΠΕ
(επιλογής)}\label{ux3baux3bfux3b9ux3bdux3c9ux3bdux3b9ux3baux3ac-ux3baux3b1ux3b9-ux3bdux3bfux3bcux3b9ux3baux3ac-ux3b8ux3adux3bcux3b1ux3c4ux3b1-ux3c4ux3c9ux3bd-ux3c4ux3c0ux3b5-ux3b5ux3c0ux3b9ux3bbux3bfux3b3ux3aeux3c2}}

\emph{Δικαιώματα και Υποχρεώσεις στην Κοινωνία της Πληροφορίας: Νομικό
και κανονιστικό πλαίσιο, ηθικά και κοινωνιολογικά ζητήματα, θέματα
κουλτούρας, δεοντολογία, ερευνητικές προεκτάσεις. Ηλεκτρονικό Έγκλημα --
κυβερνοέγκλημα. Ηλεκτρονικές Συναλλαγές και Προστασία Καταναλωτή.
Επεξεργασία προσωπικών και ευαίσθητων δεδομένων στην παροχή Διαδικτυακών
υπηρεσιών: νομικά, ηθικά, κοινωνιολογικά και τεχνολογικά ζητήματα.}

\hypertarget{ux3b1ux3bdux3b1ux3c0ux3b1ux3c1ux3acux3c3ux3c4ux3b1ux3c3ux3b7-ux3c0ux3bbux3b7ux3c1ux3bfux3c6ux3bfux3c1ux3b9ux3ceux3bd-ux3baux3b1ux3b9-ux3b3ux3bdux3ceux3c3ux3b7ux3c2-ux3b5ux3c0ux3b9ux3bbux3bfux3b3ux3aeux3c2}{%
\subsubsection{Αναπαράσταση Πληροφοριών και Γνώσης
(επιλογής)}\label{ux3b1ux3bdux3b1ux3c0ux3b1ux3c1ux3acux3c3ux3c4ux3b1ux3c3ux3b7-ux3c0ux3bbux3b7ux3c1ux3bfux3c6ux3bfux3c1ux3b9ux3ceux3bd-ux3baux3b1ux3b9-ux3b3ux3bdux3ceux3c3ux3b7ux3c2-ux3b5ux3c0ux3b9ux3bbux3bfux3b3ux3aeux3c2}}

\emph{Κύκλος Διαχείρισης της Γνώσης, Συστήματα Διαχείρισης Γνώσης,
Κύκλος Ανάπτυξης Συστημάτων Διαχείρισης Γνώσης, Αρχιτεκτονική και
Τεχνικά Χαρακτηριστικά Συστημάτων Διαχείρισης Γνώσης, Απόκτηση
Συστήματος Διαχείρισης Γνώσης, Εισαγωγή στο RDF, Κωδικοποίηση Γνώσης,
Εργαλεία και Διαδικασίες Κωδικοποίησης Γνώσης, Στρατηγικές Διαχείρισης
Γνώσης, Σημασιολογική Αναπαράσταση Γνώσης, Οργάνωση Πληροφορίας, Web
3.0, Οντολογίες και Αναπαράσταση Γνώσης Πεδίου, Φολκσονομίες, Δομημένες
Περιγραφές, Συλλογιστική, Σημασιολογικοί Κανόνες, Υπολογιστική Λογική,
Προχωρημένα θέματα γλωσσών σημασιολογίας (RDF, OWL), Θέματα αβεβαιότητας
και ασάφειας.}

\hypertarget{ux3b4ux3b9ux3b1ux3c7ux3b5ux3afux3c1ux3b9ux3c3ux3b7-ux3bcux3b5ux3b3ux3acux3bbux3bfux3c5-ux3ccux3b3ux3baux3bfux3c5-ux3b4ux3b5ux3b4ux3bfux3bcux3adux3bdux3c9ux3bd-ux3b5ux3c0ux3b9ux3bbux3bfux3b3ux3aeux3c2}{%
\subsubsection{Διαχείριση Μεγάλου Όγκου Δεδομένων
(επιλογής)}\label{ux3b4ux3b9ux3b1ux3c7ux3b5ux3afux3c1ux3b9ux3c3ux3b7-ux3bcux3b5ux3b3ux3acux3bbux3bfux3c5-ux3ccux3b3ux3baux3bfux3c5-ux3b4ux3b5ux3b4ux3bfux3bcux3adux3bdux3c9ux3bd-ux3b5ux3c0ux3b9ux3bbux3bfux3b3ux3aeux3c2}}

\emph{Τα Υπολογιστικά Νέφη (Cloud Computing). Βάσεις δεδομένων στο
διαδίκτυο: Σχεσιακές, παράλληλες και κατανεμημένες βάσεις, με έμφαση
στις τεχνολογίες κατανεμημένων συστημάτων αρχείων (HDFS), ΝοSQL (HBase,
Cassandra), graph--databases(Neo4j). Μοντέλα υπολογισμού μεγάλου όγκου
δεδομένων (MapReduce, BSP) και πλατφόρμες που τις εφαρμόζουν (Hadoop,
Hama, Spark, κλπ). Modern Data Science και η γλώσσα R. Εφαρμογές των
παραπάνω και υλοποίηση αλγορίθμων με κατανεμημένο τρόπο για επεξεργασία
μεγάλου όγκου δεδομένων. Το μάθημα αποτελεί προηγμένο εκπαιδευτικό
εργαλείο σχετικό με θεωρητικά και πρακτικά θέματα τεχνολογιών αιχμής για
την επεξεργασία δεδομένων κλίμακος. Η ύλη του μαθήματος στοχεύει στην
όσο πιο ουσιαστική εισαγωγή των φοιτητών στις βασικές αρχές, δομή και
λειτουργία των τεχνολογιών επεξεργασίας μεγάλου όγκου δεδομένων καθώς
και στα μοντέρνα εργαλεία τα οποία προσφέρονται για την υλοποίησή τους.
Ως εκ τούτου, η ύλη αναφέρεται σε προηγμένες έννοιες σχετικές με τη
διαχείριση δεδομένων μεγάλου όγκου. Εκτείνεται τόσο σε θεωρητικό
(περιγραφή συστημάτων και αρχιτεκτονικών) όσο και πρακτικό
(προγραμματιστικές γλώσσες, εργαλεία, βιβλιοθήκες) επίπεδο, προσφέροντας
γνώσεις και στους δύο άξονες. Έτσι, ο φοιτητής να έχει μία συνολική
αντίληψη των εργαλείων και μεθοδολογιών που βρίσκονται στην αιχμή της
τεχνολογίας σχετικά με τα μεγάλου όγκου δεδομένα. Εκτιμώντας τη σημασία
που έχει στις μέρες μας η δημιουργία και διαχείριση δεδομένων τόσο σε
οικονομικό όσο και κοινωνικό επίπεδο, το μάθημα λειτουργεί τόσο
μαθησιακά όσο και καθαρά εμπειρικά σχετικά με τεχνολογίες που
χρησιμοποιούνται καθημερινά από εκατομμύρια πολίτες. Τέλος, στόχο του
μαθήματος αποτελεί και η έκθεση των φοιτητών, μέσω των εργαστηριακών
ασκήσεων, σε πολλές από τις τεχνολογίες αιχμής (π.χ., NoSQL databases,
Hadoop, Spark, κλπ) κυρίως σε πρακτικό επίπεδο, με σκοπό την κατανόηση
βασικών αρχών, δυσκολίας αλλά και τριβής με έναν χώρο που αλλάζει
ραγδαία.}

\hypertarget{ux3b1ux3c3ux3c6ux3acux3bbux3b5ux3b9ux3b1-ux3bbux3bfux3b3ux3b9ux3c3ux3bcux3b9ux3baux3bfux3cd-ux3baux3b1ux3b9-ux3b5ux3c6ux3b1ux3c1ux3bcux3bfux3b3ux3ceux3bd-ux3b5ux3c0ux3b9ux3bbux3bfux3b3ux3aeux3c2}{%
\subsubsection{Ασφάλεια Λογισμικού και Εφαρμογών
(επιλογής)}\label{ux3b1ux3c3ux3c6ux3acux3bbux3b5ux3b9ux3b1-ux3bbux3bfux3b3ux3b9ux3c3ux3bcux3b9ux3baux3bfux3cd-ux3baux3b1ux3b9-ux3b5ux3c6ux3b1ux3c1ux3bcux3bfux3b3ux3ceux3bd-ux3b5ux3c0ux3b9ux3bbux3bfux3b3ux3aeux3c2}}

\emph{Επιθέσεις λογισμικού με τεχνικές υπερχείλισης μνήμης. Ανάλυση
πηγαίου κώδικα. Τεχνικές Fuzzing για αυτοματοποιημένη εύρεση λογικών
σφαλμάτων και ευπαθειών λογισμικού. Σύγχρονες τεχνικές εκμετάλλευσης
ευπαθειών λογισμικού. Ανέλιξη δικαιωμάτων σε συστήματα Linux και
Windows. Εκτίμηση αδυναμιών και έλεγχος ασφάλειας πληροφοριακών
συστημάτων. Εισαγωγή σε τεχνολογίες Web, όπως PHP, HTML, SQL,
JavaScript. Επιθέσεις Cross Site Scripting attacks (XSS) και Cross Site
Request Forgery (CSRF). Επιθέσεις SQL injection και Local file inclusion
(LFI).Σφάλματα στην αυθεντικοποίηση, διαχείριση συνόδου, μηχανισμούς
ελέγχου πρόσβασης και σε κρυπτογραφικές λύσεις. Έλεγχος ασφάλειας
εφαρμογών Διαδικτύου.}

\hypertarget{ux3b5ux3c0ux3afux3b4ux3bfux3c3ux3b7-ux3c5ux3c0ux3bfux3bbux3bfux3b3ux3b9ux3c3ux3c4ux3b9ux3baux3ceux3bd-ux3c3ux3c5ux3c3ux3c4ux3b7ux3bcux3acux3c4ux3c9ux3bd-ux3b5ux3c0ux3b9ux3bbux3bfux3b3ux3aeux3c2}{%
\subsubsection{Επίδοση Υπολογιστικών Συστημάτων
(επιλογής)}\label{ux3b5ux3c0ux3afux3b4ux3bfux3c3ux3b7-ux3c5ux3c0ux3bfux3bbux3bfux3b3ux3b9ux3c3ux3c4ux3b9ux3baux3ceux3bd-ux3c3ux3c5ux3c3ux3c4ux3b7ux3bcux3acux3c4ux3c9ux3bd-ux3b5ux3c0ux3b9ux3bbux3bfux3b3ux3aeux3c2}}

\emph{Βασικά προβλήματα αναφοράς. Υπολογιστικές αρχιτεκτονικές
υπολογιστικών συστημάτων. Αρχιτεκτονικές νέφους, συστάδων και
κατανημημένες αρχιτεκτονικές. Επισκόπηση εννοιών από την Θεωρία
Πιθανοτήτων, με έμφαση σε κατανομές τυχαίων μεταβλητών χωρίς μνήμη
(κατανομή Poisson και εκθετική κατανομή). Στοχαστικές ανελίξεις Markov.
Στασιμότητα και εργοδικότητα. Ορισμοί και βασικά πρότυπα αναμονής
(queuing models). Διαδικασίες αφίξεων και εξυπηρέτησης πελατών.
Χρησιμοποίηση εξυπηρετητή. Μέση κατάσταση ουράς αναμονής. Μέσος χρόνος
καθυστέρησης. Νόμος του Little. Ρυθμαπόδοση (throughput). Πιθανότητα
απώλειας. Διαδικασίες γεννήσεων -- θανάτων και εφαρμογές. Απλά συστήματα
αναμονής Markov M/M/1, M/M/1/K, M/M/N, M/M/N/N. Ανοικτά και κλειστά
δίκτυα ουρών αναμονής. Θεωρήματα Burke και Jackson. Εφαρμογές στην
ανάλυση επιδόσεων δικτύων μετάδοσης δεδομένων (Internet),
πολύ-επεξεργαστικών υπολογιστικών συστημάτων, πληροφοριακών συστημάτων
εξυπηρέτησης πελατών, συστημάτων διεκπεραίωσης επερωτήσεων σε βάσεις
δεδομένων και συστημάτων εξυπηρέτησης πελατών, π.χ. τραπεζικών, διοδίων,
κλπ. Ηλικία-της-Πληροφορίας στο Διαδίκτυο-των-Πραγμάτων: εισαγωγή και
ανάλυση.}

\hypertarget{ux3b5ux3c0ux3b9ux3c3ux3c4ux3aeux3bcux3b7-ux3c4ux3c9ux3bd-ux3b4ux3b5ux3b4ux3bfux3bcux3adux3bdux3c9ux3bd-ux3b5ux3c0ux3b9ux3bbux3bfux3b3ux3aeux3c2}{%
\subsubsection{Επιστήμη των Δεδομένων
(επιλογής)}\label{ux3b5ux3c0ux3b9ux3c3ux3c4ux3aeux3bcux3b7-ux3c4ux3c9ux3bd-ux3b4ux3b5ux3b4ux3bfux3bcux3adux3bdux3c9ux3bd-ux3b5ux3c0ux3b9ux3bbux3bfux3b3ux3aeux3c2}}

\emph{Περιγραφή δεδομένων με γραφήματα και πίνακες. Παρουσίαση των
βασικών στατιστικών μέτρων για τη περιγραφή δεδομένων. Προετοιμασία
Δεδομένων. Η σημασία του ελέγχου και «ξεκαθαρίσματος» των δεδομένων
(data cleaning). Εισαγωγή στις Βάσεις Δεδομένων. SQL. Εισαγωγή στην
επιβλεπόμενη μάθηση: δέντρα απόφασης, λογιστική παλινδρόμηση. Εισαγωγή
στην παλινδρόμηση: Πολλαπλή γραμμική παλινδρόμηση. Προβλέψεις. Βελτίωση
ενός μοντέλου. Τα προβλήματα της υπερ-παραμετροποίησης
(over-parametrization). Έλεγχος απόδοσης του μοντέλου. Μείωση Διαστάσεων
(Dimensionality Reduction). Η διαδικασία επιλογής χαρακτηριστικών. Η
μέθοδος των Κύριων Συνιστωσών (Principal Component Analysis) με SVD
παραγοντοποιήση μητρώων. Μη-επιβλεπόμενη μάθηση, Ανάλυση κατά συστάδες
(Clustering). Εφαρμογές και αξιολόγηση k-means. Εφαρμογή μοντέλων
Ιεραρχικού Clustering. Ημι-επιβλεπόμενη μάθηση. Εισαγωγή στα
μεταδεδομένα και στα Μεγάλα Δεδομένα (Big Data). Υπολογιστικές Μέθοδοι
για Ανάλυση Μεγάλων Δεδομένων (Hadoop και MapReduce).}

\hypertarget{ux3b5ux3beux3b5ux3b9ux3b4ux3b9ux3baux3b5ux3c5ux3bcux3adux3bdux3b1-ux3b8ux3adux3bcux3b1ux3c4ux3b1-ux3b1ux3bbux3b3ux3bfux3c1ux3afux3b8ux3bcux3c9ux3bd-ux3b5ux3c0ux3b9ux3bbux3bfux3b3ux3aeux3c2}{%
\subsubsection{Εξειδικευμένα Θέματα Αλγορίθμων
(επιλογής)}\label{ux3b5ux3beux3b5ux3b9ux3b4ux3b9ux3baux3b5ux3c5ux3bcux3adux3bdux3b1-ux3b8ux3adux3bcux3b1ux3c4ux3b1-ux3b1ux3bbux3b3ux3bfux3c1ux3afux3b8ux3bcux3c9ux3bd-ux3b5ux3c0ux3b9ux3bbux3bfux3b3ux3aeux3c2}}

\emph{Βασικές έννοιες για τους πολυνηματικούς αλγόριθμους και τη
δυναμική πολυνημάτωση. Η πολυνηματική εκδοχή του αλγόριθμου Merge Sort.
Εισαγωγή στο γραμμικό προγραμματισμό. Αναγωγή προβλημάτων σε γραμμικά
προγράμματα. Ο αλγόριθμος Simplex. Η έννοια της δυϊκότητας στον γραμμικό
προγραμματισμό. Παίγνια μηδενικού αθροίσματος. Ο διακριτός και ο ταχύς
μετασχηματισμός Fourier. Στοιχειώδεις αριθμοθεωρητικοί αλγόριθμοι. Ο
αλγόριθμος του μέγιστου κοινού διαιρέτη. Αριθμητική υπολοίπων. Το
κρυπτοσύστημα δημόσιου κλειδιού RSA. Τυχαιοποιημένοι αλγόριθμοι.
Εισαγωγή στους προσεγγιστικούς αλγόριθμους.}

\hypertarget{ux3b5ux3beux3acux3bcux3b7ux3bdux3bf-ux3b7}{%
\subsection{Εξάμηνο Η}\label{ux3b5ux3beux3acux3bcux3b7ux3bdux3bf-ux3b7}}

\hypertarget{ux3c3ux3c4ux3c1ux3b1ux3c4ux3b7ux3b3ux3b9ux3baux3ae-ux3baux3b1ux3b9-ux3b4ux3b9ux3bfux3afux3baux3b7ux3c3ux3b7-ux3c0ux3bbux3b7ux3c1ux3bfux3c6ux3bfux3c1ux3b9ux3b1ux3baux3ceux3bd-ux3c3ux3c5ux3c3ux3c4ux3b7ux3bcux3acux3c4ux3c9ux3bd-ux3baux3b1ux3c4ux3b5ux3cdux3b8ux3c5ux3bdux3c3ux3b7ux3c2-ux3c0ux3c3}{%
\subsubsection{Στρατηγική και Διοίκηση Πληροφοριακών Συστημάτων
(κατεύθυνσης
ΠΣ)}\label{ux3c3ux3c4ux3c1ux3b1ux3c4ux3b7ux3b3ux3b9ux3baux3ae-ux3baux3b1ux3b9-ux3b4ux3b9ux3bfux3afux3baux3b7ux3c3ux3b7-ux3c0ux3bbux3b7ux3c1ux3bfux3c6ux3bfux3c1ux3b9ux3b1ux3baux3ceux3bd-ux3c3ux3c5ux3c3ux3c4ux3b7ux3bcux3acux3c4ux3c9ux3bd-ux3baux3b1ux3c4ux3b5ux3cdux3b8ux3c5ux3bdux3c3ux3b7ux3c2-ux3c0ux3c3}}

\emph{Στρατηγικός Σχεδιασμός Πληροφορικής. Σημασία Πληροφορικής στους
οργανισμούς. Στρατηγικός Σχεδιασμός Π.Σ. Ευθυγράμμιση Πληροφορικής με
Επιχειρηματικές Διαδικασίες. Μοντελοποίηση και Αναδιοργάνωση
Επιχειρησιακών Διαδικασιών (Business Process Reengineering -- BPR).
Διαχείριση και αξιολόγηση πληροφοριακών πόρων. Παροχή υπηρεσιών
πληροφορικής στους οργανισμούς. Αξιολόγηση έργων πληροφορικής.}

\hypertarget{ux3b4ux3b9ux3bfux3afux3baux3b7ux3c3ux3b7-ux3b1ux3c3ux3c6ux3acux3bbux3b5ux3b9ux3b1ux3c2-ux3c0ux3bbux3b7ux3c1ux3bfux3c6ux3bfux3c1ux3b9ux3b1ux3baux3ceux3bd-ux3c3ux3c5ux3c3ux3c4ux3b7ux3bcux3acux3c4ux3c9ux3bd-ux3baux3b1ux3c4ux3b5ux3cdux3b8ux3c5ux3bdux3c3ux3b7ux3c2-ux3c0ux3c3}{%
\subsubsection{Διοίκηση Ασφάλειας Πληροφοριακών Συστημάτων (κατεύθυνσης
ΠΣ)}\label{ux3b4ux3b9ux3bfux3afux3baux3b7ux3c3ux3b7-ux3b1ux3c3ux3c6ux3acux3bbux3b5ux3b9ux3b1ux3c2-ux3c0ux3bbux3b7ux3c1ux3bfux3c6ux3bfux3c1ux3b9ux3b1ux3baux3ceux3bd-ux3c3ux3c5ux3c3ux3c4ux3b7ux3bcux3acux3c4ux3c9ux3bd-ux3baux3b1ux3c4ux3b5ux3cdux3b8ux3c5ux3bdux3c3ux3b7ux3c2-ux3c0ux3c3}}

\emph{Εννοιολογικό πλαίσιο διοίκησης ασφάλειας. Σχεδιασμός ασφάλειας:
προσεγγίσεις και ανάλυση επικινδυνότητας. Μέθοδοι και εργαλεία
λογισμικού για την ανάλυση επικινδυνότητας (CRAMM, OCTAVE, SBA
analysis). Πρότυπα διακυβέρνησης ασφάλειας. Πλαίσια, οδηγίες και
πιστοποιήσεις για την διοίκηση ασφάλειας πληροφοριακών συστημάτων (ISO
27001, ISO 27002, NIST SP 800-30, κ.ά.) Επιθεώρηση ασφάλειας
πληροφοριακών συστημάτων. Πολιτικές ασφάλειας. Σκοπός, δομή και
περιεχόμενα οργανωσιακών πολιτικών ασφάλειας. Παραδείγματα γενικευμένων
και εξειδικευμένων πολιτικών ασφάλειας. Διαχείριση Περιστατικών
Ασφάλειας. Σχεδιασμός για τη διαχείριση περιστατικών ασφάλειας, βήματα
αντιμετώπισης περιστατικών ασφάλειας. Επιχειρησιακή συνέχεια και σχέδιο
ανάκαμψης από καταστροφή. Ενημερότητα ασφάλειας. Ο Γενικός Κανονισμός
για την Προστασία Δεδομένων. Ανάλυση αντικτύπου για την προστασία
δεδομένων. Μέθοδοι και εργαλεία λογισμικού για την υλοποίηση ανάλυσης
αντικτύπου παραβίασης προσωπικών δεδομένων. Μέτρηση ασφάλειας
πληροφοριών.}

\hypertarget{ux3c3ux3b7ux3bcux3b1ux3c3ux3b9ux3bfux3bbux3bfux3b3ux3b9ux3baux3ccux3c2-ux3baux3b1ux3b9-ux3baux3bfux3b9ux3bdux3c9ux3bdux3b9ux3baux3ccux3c2-ux3b9ux3c3ux3c4ux3ccux3c2-ux3baux3b1ux3c4ux3b5ux3cdux3b8ux3c5ux3bdux3c3ux3b7ux3c2-ux3c0ux3b1ux3baux3b5}{%
\subsubsection{Σημασιολογικός και Κοινωνικός Ιστός (κατεύθυνσης
ΠΑΚΕ)}\label{ux3c3ux3b7ux3bcux3b1ux3c3ux3b9ux3bfux3bbux3bfux3b3ux3b9ux3baux3ccux3c2-ux3baux3b1ux3b9-ux3baux3bfux3b9ux3bdux3c9ux3bdux3b9ux3baux3ccux3c2-ux3b9ux3c3ux3c4ux3ccux3c2-ux3baux3b1ux3c4ux3b5ux3cdux3b8ux3c5ux3bdux3c3ux3b7ux3c2-ux3c0ux3b1ux3baux3b5}}

\emph{Ιστορικά στοιχεία. Μοντέλα και δομές πληροφορίας με στόχο την
αποδοτική διαχείριση δεδομένων του Παγκόσμιου Ιστού. Οργάνωση
πληροφορίας στον Παγκόσμιο Ιστό: semantics, οντολογίες και γλώσσες
σημασιολογίας (RDF, OWL). Η γλώσσα ερωτημάτων SPARQL. Ανοικτά
Διασυνδεδεμένα Δεδομένα. Επεξεργασία chat text, text analytics, text
mining και web sentiment analysis. Επεξεργασία δεδομένων από κοινωνικά
δίκτυα. Τεχνολογίες Web 2.0 και μηχανές αναζήτησης Ιστού. Δομή των
κοινωνικών δικτύων. Εφαρμογές πληθοπορισμού.}

\hypertarget{ux3b5ux3c0ux3b5ux3beux3b5ux3c1ux3b3ux3b1ux3c3ux3afux3b1-ux3bfux3bcux3b9ux3bbux3afux3b1ux3c2-ux3baux3b1ux3b9-ux3aeux3c7ux3bfux3c5-ux3baux3b1ux3c4ux3b5ux3cdux3b8ux3c5ux3bdux3c3ux3b7ux3c2-ux3c0ux3b1ux3baux3b5}{%
\subsubsection{Επεξεργασία Ομιλίας και Ήχου (κατεύθυνσης
ΠΑΚΕ)}\label{ux3b5ux3c0ux3b5ux3beux3b5ux3c1ux3b3ux3b1ux3c3ux3afux3b1-ux3bfux3bcux3b9ux3bbux3afux3b1ux3c2-ux3baux3b1ux3b9-ux3aeux3c7ux3bfux3c5-ux3baux3b1ux3c4ux3b5ux3cdux3b8ux3c5ux3bdux3c3ux3b7ux3c2-ux3c0ux3b1ux3baux3b5}}

\emph{Μοντελοποίηση του μηχανισμού παραγωγής ομιλίας: Μηχανισμός
παραγωγής ομιλίας, Ήχοι ομιλίας. Ψηφιακή προεπεξεργασία κειμένου
ομιλίας: Επιλογή της συχνότητας δειγματοληψίας, Ψηφιοποίηση, Βραχύχρονη
ανάλυση σήματος ομιλίας, Επιλογή μήκους πλαισίου, Προέμφαση, Επιλογή
φίλτρου ``παραθύρου'', Ρυθμός μετακίνησης πλαισίων. Ακουστικές
παράμετροι: Εξαγωγή παραμέτρων, Ακουστικές πληροφορίες διάκρισης
ομιλητών, Ενέργεια και μηδενικές διελεύσεις, Θεμελιώδης συχνότητα,
Μέθοδοι υπολογισμού τονικότητας, Φασματογράφημα, Συντονισμοί φωνητικού
καναλιού (FORMANTS), Συντελεστές γραμμικής πρόγνωσης (LPC), τράπεζα
φίλτρων, συντελεστές ανάκλασης, Cepstral Συντελεστές. Βασικές Τεχνικές
Επεξεργασίας ομιλίας. Κρυμμένα Μοντέλα Μarkov: Ορισμός και θεμελιώδεις
αλγόριθμοι. Συστήματα αναγνώρισης/κατανόησης ομιλίας, Συστήματα
Αναγνώρισης Ομιλητή. Σύνθεση ομιλίας. Ψηφιακές τεχνικές αφαίρεσης
θορύβου.}

\hypertarget{ux3b1ux3c0ux3bfux3b8ux3aeux3baux3b5ux3c2-ux3b4ux3b5ux3b4ux3bfux3bcux3adux3bdux3c9ux3bd-ux3baux3b1ux3b9-ux3b5ux3beux3ccux3c1ux3c5ux3beux3b7-ux3b3ux3bdux3ceux3c3ux3b7ux3c2-ux3b5ux3c0ux3b9ux3bbux3bfux3b3ux3aeux3c2}{%
\subsubsection{Αποθήκες Δεδομένων και Εξόρυξη Γνώσης
(επιλογής)}\label{ux3b1ux3c0ux3bfux3b8ux3aeux3baux3b5ux3c2-ux3b4ux3b5ux3b4ux3bfux3bcux3adux3bdux3c9ux3bd-ux3baux3b1ux3b9-ux3b5ux3beux3ccux3c1ux3c5ux3beux3b7-ux3b3ux3bdux3ceux3c3ux3b7ux3c2-ux3b5ux3c0ux3b9ux3bbux3bfux3b3ux3aeux3c2}}

\emph{Ιστορικά στοιχεία. Αποθήκες δεδομένων. Εισαγωγή στην εξαγωγή
γνώσης. Μεθοδολογίες και αλγόριθμοι εξόρυξης δεδομένων. Προγνωστικές και
περιγραφικές προσεγγίσεις εξόρυξης δεδομένων. Κανόνες συσχέτισης.
Κατηγοριοποίηση. Δένδρα απόφασης. k-κοντινότεροι γείτονες. Στοχαστική
κατηγοριοποίηση. Naive Bayes. Συσταδοποίηση. Μέτα-μάθηση. Αποθήκες
δεδομένων και τεχνολογίες OLAP. Προ-επεξεργασία και φιλτράρισμα
δεδομένων. Επιλογή χαρακτηριστικών. Οπτικοποίηση δεδομένων. Αξιολόγηση}

\hypertarget{ux3b7ux3bbux3b5ux3baux3c4ux3c1ux3bfux3bdux3b9ux3baux3ae-ux3b4ux3b9ux3b1ux3baux3c5ux3b2ux3adux3c1ux3bdux3b7ux3c3ux3b7-ux3b5ux3c0ux3b9ux3bbux3bfux3b3ux3aeux3c2}{%
\subsubsection{Ηλεκτρονική Διακυβέρνηση
(επιλογής)}\label{ux3b7ux3bbux3b5ux3baux3c4ux3c1ux3bfux3bdux3b9ux3baux3ae-ux3b4ux3b9ux3b1ux3baux3c5ux3b2ux3adux3c1ux3bdux3b7ux3c3ux3b7-ux3b5ux3c0ux3b9ux3bbux3bfux3b3ux3aeux3c2}}

\emph{Εισαγωγή στην Ηλεκτρονική Διακυβέρνηση. Προκλήσεις Ηλεκτρονικής
Διακυβέρνησης και Εισαγωγή στην Ηλεκτρονική Δημόσια Διοίκηση. Εσωτερικά
Πληροφοριακά Συστήματα Δημόσιας Διοίκησης. Εξωστρεφή Πληροφοριακά
Συστήματα Δημόσιας Διοίκησης. Πλαίσιο Πιστοποίησης Δημόσιων Διαδικτυακών
Τόπων. Ηλεκτρονική Δημοκρατία και Ηλεκτρονικές Προμήθειες.
Διαλειτουργικότητα Συστημάτων και Ηλεκτρονική Διακυβέρνηση.
Ανασχεδιασμός επιχειρηματικών διαδικασιών και Ηλεκτρονική Διακυβέρνηση.
Πλαίσιο Ψηφιακής Αυθεντικοποίησης και Ηλεκτρονική Διακυβέρνηση.
Καινοτόμες μορφές ηλεκτρονικής διακυβέρνησης. Μελέτες Περίπτωσης
ηλεκτρονικής διακυβέρνησης}

\hypertarget{ux3c0ux3b1ux3c1ux3acux3bbux3bbux3b7ux3bbux3bfux3c2-ux3c0ux3c1ux3bfux3b3ux3c1ux3b1ux3bcux3bcux3b1ux3c4ux3b9ux3c3ux3bcux3ccux3c2-ux3b5ux3c0ux3b9ux3bbux3bfux3b3ux3aeux3c2}{%
\subsubsection{Παράλληλος Προγραμματισμός
(επιλογής)}\label{ux3c0ux3b1ux3c1ux3acux3bbux3bbux3b7ux3bbux3bfux3c2-ux3c0ux3c1ux3bfux3b3ux3c1ux3b1ux3bcux3bcux3b1ux3c4ux3b9ux3c3ux3bcux3ccux3c2-ux3b5ux3c0ux3b9ux3bbux3bfux3b3ux3aeux3c2}}

\emph{Δομικά στοιχεία ενός υπολογιστικού συστήματος: μια ανάλυση
απόδοσης. Κρυφές μνήμες και ιεραρχίες μνημών. Παραλληλισμός σε επίπεδο
εντολών και pipelining. Παράλληλος προγραμματισμός με εντολές SSE.
Παραλληλισμός σε επίπεδο νημάτων (threads). Εισαγωγή στον προγραμματισμό
με Posix Threads. Προγραμματισμός OpenMP. Το υπολογιστικό μοντέλο GPU.
Προγραμματισμός CUDA/OpenCL.}

\hypertarget{ux3adux3beux3c5ux3c0ux3bdux3b1-ux3c0ux3b5ux3c1ux3b9ux3b2ux3acux3bbux3bbux3bfux3bdux3c4ux3b1-ux3baux3b1ux3b9-ux3b5ux3c6ux3b1ux3c1ux3bcux3bfux3b3ux3adux3c2-ux3b5ux3c0ux3b9ux3bbux3bfux3b3ux3aeux3c2}{%
\subsubsection{Έξυπνα Περιβάλλοντα και Εφαρμογές
(επιλογής)}\label{ux3adux3beux3c5ux3c0ux3bdux3b1-ux3c0ux3b5ux3c1ux3b9ux3b2ux3acux3bbux3bbux3bfux3bdux3c4ux3b1-ux3baux3b1ux3b9-ux3b5ux3c6ux3b1ux3c1ux3bcux3bfux3b3ux3adux3c2-ux3b5ux3c0ux3b9ux3bbux3bfux3b3ux3aeux3c2}}

\emph{H περιοχή της Επεξεργασίας της Πληροφορίας σε Σύγχρονες Εφαρμογές
ασχολείται με τη συλλογή, διαχείριση, οργάνωση και επεξεργασία δεδομένων
και πληροφορίας σε σύγχρονα περιβάλλοντα και εφαρμογές. Έξυπνες
εφαρμογές στις κατοικίες και στην παραγωγή. Το διαδίκτυο των πραγμάτων
(Internet of Things IoT). Physical Computing. Δημιουργία προτύπων
συστημάτων συλλογής πληροφοριών από το περιβάλλον και επέμβασης σ' αυτό,
με χρήση της πλατφόρμας Arduino.}

\hypertarget{ux3bcux3b5ux3c4ux3b1ux3b3ux3bbux3c9ux3c4ux3c4ux3b9ux3c3ux3c4ux3adux3c2-ux3b5ux3c0ux3b9ux3bbux3bfux3b3ux3aeux3c2}{%
\subsubsection{Μεταγλωττιστές
(επιλογής)}\label{ux3bcux3b5ux3c4ux3b1ux3b3ux3bbux3c9ux3c4ux3c4ux3b9ux3c3ux3c4ux3adux3c2-ux3b5ux3c0ux3b9ux3bbux3bfux3b3ux3aeux3c2}}

\emph{Εισαγωγή στη μεταγλώττιση των προγραμμάτων. Γλώσσες γενικού σκοπού
και ειδικές γλώσσες πεδίου (domain specific languages -- DSLs). Λεκτική
ανάλυση και εξαγωγή συμβόλων από πηγαίο κώδικα. Κανονικές Εκφράσεις και
η πρακτική εφαρμογή τους. Αλγόριθμοι συντακτικής ανάλυσης. Πρακτική
συντακτική ανάλυση top-down. Parsing Expression Grammars (PEGs). Πίνακες
συμβόλων και ενδιάμεσος κώδικας. Εργαλεία μεταγλώττισης: διερμηνευτές
(interpreters), συμβολομεταφραστές (assemblers), συνδέτες (linkers) και
φορτωτές (loaders).}

\hypertarget{ux3b4ux3b9ux3b1ux3c7ux3b5ux3afux3c1ux3b9ux3c3ux3b7-ux3adux3c1ux3b3ux3c9ux3bd-ux3c0ux3bbux3b7ux3c1ux3bfux3c6ux3bfux3c1ux3b9ux3baux3aeux3c2-ux3b5ux3c0ux3b9ux3bbux3bfux3b3ux3aeux3c2}{%
\subsubsection{Διαχείριση Έργων Πληροφορικής
(επιλογής)}\label{ux3b4ux3b9ux3b1ux3c7ux3b5ux3afux3c1ux3b9ux3c3ux3b7-ux3adux3c1ux3b3ux3c9ux3bd-ux3c0ux3bbux3b7ux3c1ux3bfux3c6ux3bfux3c1ux3b9ux3baux3aeux3c2-ux3b5ux3c0ux3b9ux3bbux3bfux3b3ux3aeux3c2}}

\emph{Εισαγωγή στη Δικτυωτή Ανάλυση (Έργο, δραστηριότητα,
αλληλοσυσχετίσεις δραστηριοτήτων, τοξωτά και κομβικά δίκτυα). Τα
Χαρακτηριστικά των Έργων Πληροφορικής (Χαρακτηριστικά, ιδιαιτερότητες,
κύκλος ζωής, στοιχεία κόστους, ανθρώπινο δυναμικό και εξοπλισμός).
Επίλυση Δικτύων (Αλγόριθμοι επίλυσης τοξωτών και κομβικών
δικτύων).Τεχνική PERT (Κατανομή Β, κανονική κατανομή, χρήση στατιστικών
πινάκων). Ελαχιστοποίηση Κόστους -- Μέθοδος CPM (Σχέση κόστους και
διάρκειας δραστηριότητας, αλγόριθμος ελαχιστοποίησης κόστους,
προσδιορισμός βέλτιστου χρόνου). Προγραμματισμός Δυναμικού (Μεθοδολογίες
προγραμματισμού δυναμικού, διάγραμμα Gantt, εφαρμογή heuristics, μέθοδος
εξομάλυνσης δυναμικού). Χρήση Λογισμικού για τη Διαχείριση Έργων
Πληροφορικής (Εκμάθηση και χρήση εξειδικευμένου λογισμικού (MS-Project,
MSIS)). Επίλυση Προβλημάτων και Μελετών Περίπτωσης Έργων Πληροφορικής.}

\hypertarget{ux3b5ux3b9ux3b4ux3b9ux3baux3ac-ux3b8ux3adux3bcux3b1ux3c4ux3b1-ux3b4ux3b9ux3b4ux3b1ux3baux3c4ux3b9ux3baux3aeux3c2-ux3c4ux3b7ux3c2-ux3c0ux3bbux3b7ux3c1ux3bfux3c6ux3bfux3c1ux3b9ux3baux3aeux3c2-ux3b5ux3c0ux3b9ux3bbux3bfux3b3ux3aeux3c2}{%
\subsubsection{Ειδικά θέματα Διδακτικής της Πληροφορικής
(επιλογής)}\label{ux3b5ux3b9ux3b4ux3b9ux3baux3ac-ux3b8ux3adux3bcux3b1ux3c4ux3b1-ux3b4ux3b9ux3b4ux3b1ux3baux3c4ux3b9ux3baux3aeux3c2-ux3c4ux3b7ux3c2-ux3c0ux3bbux3b7ux3c1ux3bfux3c6ux3bfux3c1ux3b9ux3baux3aeux3c2-ux3b5ux3c0ux3b9ux3bbux3bfux3b3ux3aeux3c2}}

\emph{Η Πληροφορική στην Εκπαίδευση. Οι ΤΠΕ ως µέσο γνώσης, έρευνας και
µάθησης στα διάφορα γνωστικά αντικείµενα. Βασικές έννοιες και
χρησιµοποιούµενη ορολογία του τοµέα της ∆ιδακτικής της Πληροφορικής.
Παραδοσιακές διδακτικές προσεγγίσεις και προσεγγίσεις που βασίζονται σε
σύγχρονες θεωρίες µάθησης, µαθησιακές δυσκολίες σε βασικές έννοιες της
Πληροφορικής, παραδείγµατα από σχέδια µαθήµατος και δραστηριότητες.}
